%%%%%%%%%%%%%%%%%%%% author.tex %%%%%%%%%%%%%%%%%%%%%%%%%%%%%%%%%%%
%
% sample root file for your "contribution" to a contributed volume
%
% Use this file as a template for your own input.
%
%%%%%%%%%%%%%%%% Springer %%%%%%%%%%%%%%%%%%%%%%%%%%%%%%%%%%


% RECOMMENDED %%%%%%%%%%%%%%%%%%%%%%%%%%%%%%%%%%%%%%%%%%%%%%%%%%%
\documentclass[graybox]{svmult}

% choose options for [] as required from the list
% in the Reference Guide

\usepackage{type1cm}        % activate if the above 3 fonts are
                            % not available on your system
%
\usepackage{makeidx}         % allows index generation
\usepackage{graphicx}        % standard LaTeX graphics tool
                             % when including figure files
\usepackage{multicol}        % used for the two-column index
\usepackage[bottom]{footmisc}% places footnotes at page bottom


\usepackage{newtxtext}       % 
\usepackage{newtxmath}       % selects Times Roman as basic font

\usepackage{url}

%%Nirav added
\newenvironment{spmatrix}[1]
 {\def\mysubscript{#1}\mathop\bgroup\begin{pmatrix}}
 {\end{pmatrix}\egroup_{\textstyle\mathstrut\mysubscript}}
\DeclareMathOperator{\Tr}{Tr}
\DeclareMathOperator{\spn}{span}
\usepackage{bm}
\usepackage{tikz}
\usepackage{wrapfig}
\usepackage[export]{adjustbox}
\usepackage{subcaption}
\captionsetup{compatibility=false}
%%Nirav added over

% see the list of further useful packages
% in the Reference Guide

\makeindex             % used for the subject index
                       % please use the style svind.ist with
                       % your makeindex program

%%%%%%%%%%%%%%%%%%%%%%%%%%%%%%%%%%%%%%%%%%%%%%%%%%%%%%%%%%%%%%%%%%%%%%%%%%%%%%%%%%%%%%%%%

\begin{document}

\title{Discontinuous Galerkin model order reduction of geometrically parametrized Stokes flow}
% Use \titlerunning{Short Title} for an abbreviated version of
% your contribution title if the original one is too long
\author{Nirav Vasant Shah, Martin Hess and Gianluigi Rozza}
% Use \authorrunning{Short Title} for an abbreviated version of
% your contribution title if the original one is too long
\institute{Nirav Vasant Shah \at Scuola Internazionale Superiore di Studi Avanzati - via Bonomea, 265 - 34136 Trieste ITALY, \email{snirav@sissa.it}
\and Martin Hess \at Scuola Internazionale Superiore di Studi Avanzati - via Bonomea, 265 - 34136 Trieste ITALY \email{martin.hess@sissa.it}}
%
% Use the package "url.sty" to avoid
% problems with special characters
% used in your e-mail or web address
%
\maketitle

\abstract*{Each chapter should be preceded by an abstract (no more than 200 words) that summarizes the content. The abstract will appear \textit{online} at \url{www.SpringerLink.com} and be available with unrestricted access. This allows unregistered users to read the abstract as a teaser for the complete chapter.
Please use the 'starred' version of the \texttt{abstract} command for typesetting the text of the online abstracts (cf. source file of this chapter template \texttt{abstract}) and include them with the source files of your manuscript. Use the plain \texttt{abstract} command if the abstract is also to appear in the printed version of the book.}

\abstract{The present work focuses on geometrical parametrization and reduced order modeling of Stokes flow. The importance of Stokes flow, advantages of discontinuous Galerkin method are introduced first. We also discuss the concept of geometric parametrization and its application along with importance of reduced order model technique.  The full order model is based on discontinuous Galerkin method interior penalty formulation. The concepts of broken Sobolev spaces, relevant norms, jump and mean operator are introduced. The weak formulation is derived based in suitable space to obtain the full order model. We then introduce the concept of geometric parametrization. The operators are transformed from fixed domain to parameter dependent domain by exploring affine parameter dependence which results in efficient assembly of system matrix. Thereafter, proper orthogonal decomposition is applied to obtain basis for function space for reduced order model. By using Galerkin projection the linear system to be solved is projected onto reduced space. During the process, offline-online decomposition is used to separate computation of expensive parameter independent part and fast parameter independent part. Finally the technique is applied to test problem. The numerical outcomes presented include the experimental error analysis, eigenvalue computation and measurement of online simulation time. \cite{psysoc-journal}}

\section{Introduction}
\label{introduction}

The subject of mathematical applications in fluid mechanics starts with one of the variants of the Navier-Stokes equations, such as the Stokes equation. Almost all processes of fluid mechanics require considerations related to the Navier-Stokes equations. Navier-Stokes equation is non-linear, characterizing flow fluctuations. However, in case of laminar flow, i.e. when fluctuations are negligible, this linearized form of the Navier-Stokes equation is the Stokes equation.

Discontinuous Galerkin method (DGM) has found traction as numerical method for elliptic problems \textbf{pereire reference} as well as hyperbolic problems \textbf{Book on compressible flow reference}. This is due to its several advantages over Finite Element Method (FEM) and Finite Volume Method (FVM). In fact, DG method is considered as combination of FEM and FVM. DGM uses polynomial approximation of suitable degree providing higher accuracy as well as allows discontinuity at the interface, by the concept of numerical flux, allowing greater flexibility. This fact makes DGM naturally attractive to problems such as shock capturing due to presence of steep gradients or discontinuities. Additionally, since the Dirichlet conditions are applied as boundary penalty, it avoids necessity to work with subspace of FEM. Several variants of DGM exist based on computational advantages such as sparsity pattern or extension of computational stencil, complexity of numerical implementation etc.

Geometric parametrization has emerged as important application of Parametric Partial Differential Equations (PPDEs) and as alternative to shape optimization. The concept of geometric parametrization allows to transfer operator evaluated on one domain to another domain efficiently. For linear equations, this means exploiting affine parameter dependence as will be shown in later section. Model Order Reduction (MOR) on the other hand allows reducing the size of the system to be solved and working with the smaller system containing only dominant components and discarding the non-dominant components. It is pertinent to mention that identifying "dominant" components is critical to the success of model order reduction strategy. Optimization of engineering components using geometric parametrization combined with MOR for PPDEs has given quite useful results in the fields such as mechanical, naval and aeronautic designs. Also, the  faster computations obtained by MOR has helped in many query context, real time computation and quick transfer of computational results to industrial problems.

In the present work, we first introduce Discontinuous Galerkin Interior Penalty Method (DG-IPM). We subsequently introduce notion of parametrization characterizing geometry of the domain under consideration, exploit affine parameter dependence and its application in the context of offline-online decomposition. We then apply Proper Orthogonal Decomposition (POD) for constructing reduced basis space and apply Galerkin projection to project the system of equations on the space constructed by POD. Finally we present a test problem to demonstrate the introduced method and present numerical result.
 
\section{Geometric parametrization}\label{geometric_parametrization_section}

Consider domain $\Omega = \Omega(\mu) \in \mathbb{R}^d$ as open bounded domain. The parameter set $\mu \in \mathbb{P}$, where $\mathbb{P}$ is parameter space, completely characterizes the domain. Also, consider a parameter set $\bar{\mu} \in \mathbb{P}$, as the known parameter set and $\Omega(\bar{\mu})$ as the reference domain, whose configuration is completely known. The mapping $\bm{F}(\cdot,\mu) : \Omega(\bar{\mu}) \rightarrow \Omega(\mu)$ links reference domain and parametrized domain. Divide the domain $\Omega(\mu)$ into $n_{su}$ subdomains such that $\Omega(\mu) = \bigcup\limits_{i=1}^{n_{su}} \Omega_i(\mu) \ , \ \Omega_i(\mu) \bigcap \Omega_j(\mu) = \emptyset \ , \ \text{for} \ i \neq j$. The boundary of $\Omega(\mu)$, $\partial \Omega(\mu)$ is divided into Neumann boundary $\Gamma_N(\mu)$ and Dirichlet boundary $\Gamma_D(\mu)$ i.e. $\partial \Omega(\mu) = \Gamma_N(\mu) \cup \Gamma_D(\mu)$.

In the case of affine transformation, $\bm{F}$ is of the form,
\begin{gather}\label{affine_F}
x = \bm{F}(\hat{x},\mu) = \bm{G}_F(\mu)\hat{x} + c_F(\mu) \ ; \forall x \in \Omega \ , \ \hat{x} \in \hat{\Omega} \ .
\end{gather}

The inverse map $\bm{T}$ is expressed in the form,
\begin{gather}\label{affine_T}
\hat{x} = \bm{T}(x,\mu) = \bm{G}_T(\mu)x + c_T(\mu) \ ; \forall x \in \Omega \ , \ \hat{x} \in \hat{\Omega} \ .
\end{gather}

\section{Discontinuous Galerkin formulation}
\label{DG_formulation}

Each subdomain is divided into $N_{el}$ number of triangular elements $\tau_k$ such that $\Omega = \bigcup\limits_{k=1}^{N_{el}} \tau_k$. The triangulation $\mathcal{T}$ is the set of all triangular elements i.e. $\mathcal{T} = \lbrace \tau_k \rbrace_{k=1}^{N_{el}}$. The internal boundary $\Gamma = \lbrace \partial \tau_k \rbrace_{k=1}^{N_{el}} \backslash \partial \Omega$. We represent $\overrightarrow{n}$ as the outward pointing normal to an edge of element.

The Stokes's equation in strong form can be stated as,
\begin{gather}
-\nu \Delta \overrightarrow{u} + \nabla p = \overrightarrow{f} \ , \ \text{in } \Omega \ , \\
\nabla \cdot \overrightarrow{u} = 0 \ , \\
\overrightarrow{u} = \overrightarrow{u}_D \ , \ \text{on } \Gamma_D \ , \\
-p \overrightarrow{n} + \nu \overrightarrow{n} \cdot \nabla \overrightarrow{u} = \overrightarrow{t} \ , \ \text{on } \Gamma_N \ .
\end{gather}

The vector variable velocity $\overrightarrow{u}$ and scalar pressure $p$ are the unknowns. $\nu$ is the material property known as kinematic viscosity. Vector $\overrightarrow{f}$ is external force term or source term. $\overrightarrow{u}_D$ is the Dirichlet velocity and $\overrightarrow{t}$ is the Neumann value.

Before introducing weak form let us introduce broken Sobolev spaces for variables.

The space for velocity is 
\begin{equation} \label{velocity_test}
\mathbb{V} = \lbrace \overrightarrow{\phi} \in (L^2(\mathcal{T}))^d | \ \overrightarrow{\phi} \in (P^D(\tau_k))^d \ , \ \tau_k \in \mathcal{T} \rbrace \ .
\end{equation}
The space for pressure is 
\begin{equation} \label{pressure_test}
\mathbb{Q} = \lbrace \psi \in (L^2(\mathcal{T})) | \ \psi \in (P^{D-1}(\tau_k)) \ , \ \tau_k \in \mathcal{T} \rbrace \ .
\end{equation}
Here, $P^D(\tau_k)$ denotes space of polynomials of degree at most $D \geq 2$ over $\tau_k$.

\begin{wrapfigure}{r}{5.5cm}
\begin{tikzpicture}
\draw (0,0) node[anchor=north]{${}$}
  -- (4,0) node[anchor=north]{${}$}
  -- (0,4) node[anchor=south]{${}$}
  -- cycle;
\draw (4,4) node[anchor=north]{${}$}
  -- (4,0) node[anchor=north]{${}$}
  -- (0,4) node[anchor=south]{${}$}
  -- cycle;
\draw[->] (2,2)--(2.5,2.5)node[label={[xshift=0.4cm, yshift=-0.7cm]${\overrightarrow{n}^-}$}]{};
\draw[->] (2,2)--(1.5,1.5)node[label={[xshift=0cm, yshift=0cm]${\overrightarrow{n}^+}$}]{};
\node at (1,1){$\mathcal{T^-}$};
\node at (3,3){$\mathcal{T^+}$};
\node at (1.7,3){$\overrightarrow{\phi}^+,\psi^+$};
\node at (2.3,1){$\overrightarrow{\phi}^-,\psi^-$};
\end{tikzpicture}
\caption{Element self (+) and neighbouring element (-)}
\label{fig:Self_neighbour}
\end{wrapfigure}
In order to approximate the numerical flux we need the concept of Jump and Average operator. The superscript $+$ refers to quantity in the element itself and the superscript $-$ refers to quantity in the neighboring element (Figure \ref{fig:Self_neighbour}).

The presence of normal vector $\overrightarrow{n}$ in jump and average operator introduced below allows symmetric formulation and also ensures that jump of a vector is vector and jump of a scalar is scalar.
\vspace{\baselineskip}
\vspace{\baselineskip}
\vspace{\baselineskip}
\begin{itemize}
\item For vector quantity $\overrightarrow{\phi}$:
\begin{itemize}
\item Jump operator: 
$\left[\overrightarrow{\phi} \cdot \overrightarrow{n}\right] = \overrightarrow{\phi}^+ \cdot \overrightarrow{n}^+ + \overrightarrow{\phi}^- \cdot \overrightarrow{n}^-$ on $\Gamma$, $\left[\overrightarrow{\phi} \cdot \overrightarrow{n}\right] = \overrightarrow{\phi} \cdot \overrightarrow{n}$ on $\partial \Omega$.
\item Average operator:
$\lbrace \overrightarrow{\phi} \rbrace = \frac{\overrightarrow{\phi}^+ + \overrightarrow{\phi}^-}{2}$ on $\Gamma$, $\lbrace \overrightarrow{\phi} \rbrace = \overrightarrow{\phi}$ on $\partial \Omega$.
\end{itemize}
\item For scalar quantity $\psi$:
\begin{itemize}
\item Jump operator:
$\left[\psi\right] = \psi^+ + \psi^-$ on $\Gamma$, $\left[\psi\right] = \psi$ on $\partial \Omega$.
\item Average operator:
$\lbrace \psi \rbrace = \frac{\psi^+ + \psi^-}{2}$ on $\Gamma$, $\lbrace \psi \rbrace = \psi$ on $\partial \Omega$. 
\end{itemize}
\end{itemize}

The weak form of Stokes equation is as follow,
\begin{gather}\label{stokes_weak_ch3}
a_{IP}(\overrightarrow{u},\overrightarrow{\phi}) + b(\overrightarrow{\phi},p) + \left( \lbrace p \rbrace,[\overrightarrow{n} \cdot \overrightarrow{\phi}] \right)_{\Gamma \cup \Gamma_D} = l_{IP}(\overrightarrow{\phi}) \ , \\
a_{IP}(\overrightarrow{u},\overrightarrow{\phi}) = \left( \nabla \overrightarrow{u}, \nabla \overrightarrow{\phi} \right) + C_{11} \left( [\overrightarrow{u}],[\overrightarrow{\phi}] \right)_{\Gamma \cup \Gamma_D} - \nu \left( \lbrace \nabla \overrightarrow{u}\rbrace ,[\overrightarrow{n} \otimes \overrightarrow{\phi}] \right)_{\Gamma \cup \Gamma_D} - \nu \left( [\overrightarrow{n} \otimes \overrightarrow{u}], \lbrace \nabla \overrightarrow{\phi} \rbrace \right)_{\Gamma \cup \Gamma_D} \ , \\
b(\phi,\psi) = -\int_{\mathcal{T}} \psi \nabla \cdot \overrightarrow{\phi} \ , \\
l_{IP}(\overrightarrow{\phi}) = \left( \overrightarrow{f},\overrightarrow{\phi} \right) + \left( \overrightarrow{t},\overrightarrow{\phi} \right)_{\Gamma_N} + C_{11} \left(\overrightarrow{u}_D,\overrightarrow{\phi}\right)_{\Gamma_D} - \left( \overrightarrow{n} \otimes \overrightarrow{u}_D, \nu \nabla \overrightarrow{\phi} \right)_{\Gamma_D} \ .
\end{gather}

The penalty paramter $C_{11}>0$ in $a_{IP}(\overrightarrow{u},\overrightarrow{\phi})$ is an empirical constant to be kept large enough to maintain coercivity of bilinear form.

The weak for of continuity equation is as follow,
\begin{equation}\label{contiuity_weak_ch3}
\begin{split}
b(\overrightarrow{u},\psi) + ({\psi},[\overrightarrow{n} \cdot \overrightarrow{u}])_{\Gamma \cup \Gamma_D} = (\psi,\overrightarrow{n} \cdot \overrightarrow{u}_D)_{\Gamma_D} \ .
\end{split}
\end{equation}

In matrix form system of equations can be written as, 
\begin{equation} \label{Stokes_matrix_ch3}
\begin{spmatrix}{\textrm{Stiffness matrix}}
    A & B \\
    B^T & 0
\end{spmatrix}
\begin{spmatrix}{\textrm{Solution vector}}
    U \\
    P
\end{spmatrix}
=
\begin{spmatrix}{\textrm{Right hand side (Known)}}
    F_1  \\
    F_2
\end{spmatrix}
\textrm{.}
\end{equation}

Here, $A_{ij} = a_{IP} (\overrightarrow{\phi}_i,\overrightarrow{\phi}_j)$, $B_{ij} = b(\overrightarrow{\phi}_i,\psi_j) + \left( \lbrace \psi_j \rbrace , [n \cdot \phi_i]\right)_{\Gamma \cup \Gamma_D}$, $F_1 = l_{IP}(\overrightarrow{\phi}_i)$ and $F_2 = \left( \psi_j,\overrightarrow{n} \cdot \overrightarrow{u}_D \right)_{\Gamma_D}$.

\section{Affine expansion}

We evaluate and solve the Stokes equation weak formulation on reference domain $\Omega{\bar{\mu}}$. Given a parameter set $\mu \neq \bar{\mu}$ we need to evaluate the linear systems of equation \eqref{Stokes_matrix_ch3} on new domain $\Omega(\mu)$. To accomplish this we use affine expansion using linear nature of equation and diving $\Omega(\bar{\mu})$ into triangular subdomains $\Omega_i(\bar{\mu}) \ , \ i = \lbrace 1,2,\ldots,n_{su} \rbrace$ as explained earlier in the section geometric parametrization [Section \ref{geometric_parametrization_section}]. The transformation of operators in affine is standard in literature \textbf{ADD affine expansion literature} and only two relevant section as specific to DG formulation will be mentioned here as below.

In order to transfer the terms containing jump and average operator following approach is used in present analysis.
\begin{equation}\label{jump_average_term_split}
\begin{split}
\left(\lbrace \nabla \overrightarrow{\phi} \rbrace , \left[ \overrightarrow{n} \otimes \overrightarrow{\phi}  \right]  \right) = \left( \nabla \overrightarrow{\phi}^+ , \overrightarrow{n}^+ \otimes \overrightarrow{\phi}^+ \right) + \left( \nabla \overrightarrow{\phi}^+ , \overrightarrow{n}^- \otimes \overrightarrow{\phi}^- \right) + \left( \nabla \overrightarrow{\phi}^- , \overrightarrow{n}^+ \otimes \overrightarrow{\phi}^+ \right) + \left( \nabla \overrightarrow{\phi}^- , \overrightarrow{n}^- \otimes \overrightarrow{\phi}^- \right) \ .
\end{split}
\end{equation}
Each term on right hand side of equation \eqref{jump_average_term_split} can now be transformed using affine map.

The coercivity term $C_{11}\left( [\overrightarrow{\phi},\overrightarrow{u}] \right)_{\Gamma \cup \Gamma_D}$ is not transformed but used as evaluated on reference domain $\Omega(\bar{\mu})$. The affine transformation is given by,
\begin{equation}
\begin{split}
C_{11}\left( [\overrightarrow{\phi}(\hat{x}),\overrightarrow{u}(\hat{x})] \right)_{\Gamma(\mu) \cup \Gamma_D(\mu)} = C_{11} \alpha \left( [\overrightarrow{\phi}(\bm{F}(\hat{x})),\overrightarrow{u}(\bm{F}(\hat{x}))] \right)_{\Gamma(\bar{\mu}) \cup \Gamma_D(\bar{\mu})} \ , \\
\alpha = \frac{meas\left( \Gamma(\mu) \cup \Gamma_D(\mu)\right)}{meas\left( \Gamma(\bar{\mu}) \cup \Gamma_D(\bar{\mu})\right)} \ , \ \hat{x} \in \Omega(\bar{\mu}) \ , \ x \in \Omega(\mu) \ .
\end{split}
\end{equation}
Since, $C_{11}$ is empirical coefficient replacing $C_{11} \alpha$ with $C_{11}$ will not change the formulation as long as coercivity of $a_{IP}$ over parameter space $\mathbb{P}$ is maintained. In the present analysis, $C_{11}$ is not calculated exactly but only order of magnitude required for $C_{11}$ is estimated and $C_{11}$ is kept one magnitude larger as safeguard against round-off errors. As long as the domain $\Omega(\mu)$ is not deformed much compared to its reference configuration $\Omega(\bar{\mu})$, $C_{11}$ and $C_{11}\alpha$ will have the same order of magnitude. However, large deformations are anyhow not favorable as it will lead to bad mesh quality and in turn, will lead to poor DGM-approximation.

\section{Reduced basis method}

\subsection{Snapshot proper orthogonal decomposition}\label{POD_section}

We present now snapshot proper orthogonal decomposition method. Here, ``snapshot" means solution calculated by discontinuous Galerkin method. We calculate solution based on $\mu_n, n \in \lbrace 1,....,n_s \rbrace$ i.e. $n_s$ snapshots are generated. We also introduce inner product matrices $\bm{M}_v \in \mathbb{R}^{\overrightarrow{u}_{ndofs} \times \overrightarrow{u}_{ndofs}}$ and $\bm{M}_p \in \mathbb{R}^{p_{ndofs} \times p_{ndofs}}$, where $\overrightarrow{u}_{ndofs}$ and $p_{ndofs}$ are number of velocity and pressure degrees of freedom respectively.
\begin{gather*}
\bm{M}_v = \int_{\Omega} \overrightarrow{\phi}_i \cdot \overrightarrow{\phi}_j + \sum_{k=1}^{N_{el}} \int_{\tau_k} \nabla \overrightarrow{\phi}_i : \nabla \overrightarrow{\phi}_j \ , \ i,j = 1, \ldots, \overrightarrow{u}_{ndofs} \ , \\
\bm{M}_p = \int_{\Omega} \psi_i \psi_j \ , \ i,j = 1, \ldots, p_{ndofs} \ .
\end{gather*}

We also introduce matrices storing velocity snapshots $\bm{S}_v$ and storing pressure snapshots $\bm{S}_p$. We discuss the method only for velocity snapshots. The method is similar for pressure snapshots. We note the size of matrices, useful for matrix operations presented hereafter.

\begin{gather*}
\bm{S}_v \in \mathbb{R}^{\overrightarrow{u}_{ndofs} \times n_s} \ , \ \bm{S}_p \in \mathbb{R}^{p_{ndofs} \times n_s} \ , \\
\bm{M}_v \in \mathbb{R}^{\overrightarrow{u}_{ndofs} \times \overrightarrow{u}_{ndofs}} \ ,\ \bm{M}_p \in \mathbb{R}^{p_{ndofs} \times p_{ndofs}} \ .
\end{gather*}

\subsection{Spectral decomposition of snapshots}\label{spectral_decomposition_section}

We denote the dimension of reduced basis as $N$ and assert that $N << n_s$. We now perform the spectral decomposition of $\bm{S}_v^T \bm{M}_v \bm{S}_v$,

\begin{equation}
\bm{S}_v^T \bm{M}_v \bm{S}_v = \bm{V} \bm{\Theta} \bm{V}^T \ .
\end{equation}

The columns of $V$ are eigenvectors and $\Theta$ has eigenvalues $\theta_i \ , \ 1 \leq i \leq n_s$ such that,
\begin{equation}
\Theta_{ij} = \theta_i \delta_{ij} \ .
\end{equation}

We also note that $\theta_i > 0$ and $\theta_1 \geq \theta_2 \geq ... \geq \theta_{n_s}$ i.e. the eigenvalues are in sorted order. We form the reduced basis by linear combination of the snapshot vector,
\begin{equation}
\bm{B}_v = \bm{S}_v \bm{A} \ , \ \bm{A} \in \mathbb{R}^{n_s \times N} \ .
\end{equation}

Here, $\bm{B}_v$ is defined such that, if $\overrightarrow{\phi} \in \mathbb{R}^{n \times d} \ , \ \bm{B}_v \in \mathbb{R}^{n \times N}$, the reduced basis for velocity $\overrightarrow{\phi}_N \in \mathbb{R}^{d \times N}$ is formed by,
\begin{equation}
\overrightarrow{\phi}_N = \overrightarrow{\phi}^T \bm{B}_v \ .
\end{equation}

Considering orthonormality of reduced basis $\overrightarrow{\phi}_N$ with respect to inner product $\bm{M}_v$,
\begin{equation}
<\overrightarrow{\phi}_N,\overrightarrow{\phi}_N>_{\bm{M}_v} = \bm{B}_v^T \bm{M}_v \bm{B}_v = I \ .
\end{equation}

Considering above orthonormality, we express matrix $\bm{A}$ as,
\begin{equation}
\bm{A} = \bm{V} \bm{\Theta}^{-\frac{1}{2}} \bm{R} \ , \ \bm{R} \in \mathbb{R}^{n_s \times N} \ , \ \bm{R}^T \bm{R} = \bm{I} \ .
\end{equation}
where, $I$ is identity matrix of suitable size.

We set now $\bm{R}$ as,
\begin{equation}
\bm{R} = [\bm{I}_{N \times N} ; \bm{0}_{(n_s \times N)}] \ \text{and accordingly} \ \bm{B}_v = \bm{S}_v \bm{V} \bm{\Theta}^{-\frac{1}{2}} \bm{R} \ .
\end{equation}

\subsection{Galerkin reduced basis formulation}\label{Galerkin_section}

We now present the reduced bilinear form as,

\begin{equation} \label{stokes_equation_parameter}
a(u_N,\phi_N;\mu) + b(p_N,\phi_N;\mu) = f_1(\phi_N,\mu) \textrm{,}
\end{equation}

\begin{equation} \label{continuity_equation_parameter}
b(u_N,\psi_N;\mu) = f_2(\psi_N,\mu) \textrm{.}
\end{equation}

In discrete form, we form reduced equation as,

\begin{equation} \label{Stokes_matrix_reduced}
\begin{spmatrix}{\tilde{K}}
    \bm{B}_v^T \bm{A}(\mu) \bm{B}_v & \bm{B}_v^T \bm{B}(\mu) \bm{B}_v \\
    \bm{B}_p^T \bm{B}(\mu)^T \bm{B}_v & \bm{0}
\end{spmatrix}
\begin{spmatrix}{\zeta}
    U_N \\
    P_N
\end{spmatrix}
=
\begin{spmatrix}{\tilde{F}}
    \bm{B}_v^T F_1(\mu)  \\
    \bm{B}_p^T F_2(\mu)
\end{spmatrix} \ ,
\end{equation}
and accordingly we solve following variational form for reduced degrees of freedom $\zeta$,
\begin{equation}
\tilde{\bm{K}} \zeta = \tilde{F} \ ,
\end{equation}
and calculate reduced solutions $\overrightarrow{u}_N$ and $p_N$ as,
\begin{equation}
\overrightarrow{u}_N = \bm{B}_v U_N \ , \ p_N = \bm{B}_p P_N
\end{equation}

\section{Offline-online procedure}

The offline-online procedure is used to separate computationally intensive parameter independent offline procedure and faster parameter dependent online procedure \textbf{cite CRBM}. During the offline phase $n_s$ snapshots are computed and reduced basis spaces $\bm{B}_v$ and $\bm{B}_p$ are created. The offline procedure is outlined in section \ref{POD_section} and section \ref{spectral_decomposition_section}. During the online phase the systems of equations are projected on reduced space using Galerkin projection, the smaller systems of equation obtained by Galerkin projection is solved and the reduced basis solution is computed as outlined in section \ref{Galerkin_section}. The parameter dependent matrices in equation \eqref{Stokes_matrix_reduced} are evaluated by using affine decomposition.

\section{Numerical example}

We perform the POD-Galerkin method as mentioned in section \ref{POD_section} - section \ref{Galerkin_section}. The boundary ${x=0}$ is Dirichlet boundary with inflow velocity at point $(0,y)$ as $u = (y(1-y), 0)$. The boundary ${x = 1}$ is a Neumann boundary with zero Neumann value i.e. $t = (0, 0)$. Other boundaries are Dirichlet boundary with no slip condition. The source term is $f = (0,0)$.

The geometric parameters were coordinates of tip of the obstacle on bottom edge.
\begin{gather*}
(x,y) = (\mu_1,\mu_2) \ , \ (x_{ref},y_{ref}) = (0.5,0.3) \ ,
\end{gather*}
with $(x_{ref},y_{ref})$ as the reference parameters.

The training set was generated by random generation of $100$ parameters between the interval $[0.4,0.6] \times [0.4,0.6]$. The test set contained $10$ parameters between the interval $[0.4,0.6] \times [0.4,0.6]$.

\begin{figure}%[t!] % "[t!]" placement specifier just for this example
\begin{subfigure}{\textwidth}
\includegraphics[width=\linewidth]{offline_pressure_at_43_36.jpg}
\caption{Velocity $x-$direction DG solution} \label{vel_x_dg}
\end{subfigure}\hspace*{\fill}
\begin{subfigure}{0.31\textwidth}
\includegraphics[width=\linewidth]{offline_pressure_at_43_36.jpg}
\caption{Velocity $x-$direction RB solution} \label{vel_x_rb}
\end{subfigure}
\begin{subfigure}{0.31\textwidth}
\includegraphics[width=\linewidth]{offline_pressure_at_43_36.jpg}
\caption{Pressure RB solution} \label{pre_rb}
\end{subfigure}

\begin{subfigure}{0.31\textwidth}
\includegraphics[width=\linewidth]{offline_pressure_at_43_36.jpg}
\caption{Velocity $y-$direction DG solution} \label{vel_y_dg}
\end{subfigure}\hspace*{\fill}
\begin{subfigure}{0.31\textwidth}
\includegraphics[width=\linewidth]{offline_pressure_at_43_36.jpg}
\caption{Velocity $y-$direction RB solution} \label{vel_y_rb}
\end{subfigure}
\begin{subfigure}{0.31\textwidth}
\includegraphics[width=\linewidth]{offline_pressure_at_43_36.jpg}
\caption{Pressure RB solution} \label{pre_rb}
\end{subfigure}

\begin{subfigure}{0.31\textwidth}
\includegraphics[width=\linewidth]{offline_pressure_at_43_36.jpg}
\caption{Pressure DG solution} \label{pre_dg}
\end{subfigure}\hspace*{\fill}
\begin{subfigure}{0.31\textwidth}
\includegraphics[width=\linewidth]{offline_pressure_at_43_36.jpg}
\caption{Pressure RB solution} \label{pre_rb}
\end{subfigure}
\begin{subfigure}{0.31\textwidth}
\includegraphics[width=\linewidth]{offline_pressure_at_43_36.jpg}
\caption{Pressure RB solution} \label{pre_rb}
\end{subfigure}
\caption{DG and RB solution $[\mu_x \ \mu_y] = [0.43 \ 0.36]$} 
\label{dg_rb_solution_43_36}
\end{figure}

\begin{figure}
  \includegraphics[width=\linewidth]{size_vs_maximum_reduced_basis_velocity_error_semilog.jpg}
  \caption{Size of reduced basis for velocity vs Online simulation time (semilog scale)} 
\label{online_simulation_time}
\end{figure}

\begin{figure}
  \includegraphics[width=\linewidth]{size_vs_maximum_reduced_basis_velocity_error_semilog.jpg}
  \caption{Size of reduced basis for velocity vs Online simulation time (semilog scale)} 
\label{online_simulation_time}
\end{figure}

\begin{figure}
  \includegraphics[width=\linewidth]{size_vs_maximum_reduced_basis_velocity_error_semilog.jpg}
  \caption{Size of reduced basis for velocity vs Online simulation time (semilog scale)} 
\label{online_simulation_time}
\end{figure}

\begin{figure}
  \includegraphics[width=\linewidth]{size_vs_maximum_reduced_basis_velocity_error_semilog.jpg}
  \caption{Size of reduced basis for velocity vs Online simulation time (semilog scale)} 
\label{online_simulation_time}
\end{figure}

\textbf{ADD : experimental results: Eigenvalue drop, error vs basis function, RB,DG Solution and error at one parameter value}

\begin{enumerate}
\item{Livelihood and survival mobility are oftentimes coutcomes of uneven socioeconomic development.}
\begin{enumerate}
\item{Livelihood and survival mobility are oftentimes coutcomes of uneven socioeconomic development.}
\item{Livelihood and survival mobility are oftentimes coutcomes of uneven socioeconomic development.}
\end{enumerate}
\item{Livelihood and survival mobility are oftentimes coutcomes of uneven socioeconomic development.}
\end{enumerate}


\subparagraph{Subparagraph Heading} In order to avoid simply listing headings of different levels we recommend to let every heading be followed by at least a short passage of text. Use the \LaTeX\ automatism for all your cross-references and citations as has already been described in Sect.~\ref{sec:2}, see also Fig.~\ref{fig:2}.

For unnumbered list we recommend to use the \verb|itemize| environment -- it will automatically be rendered in line with the preferred layout.

\begin{itemize}
\item{Livelihood and survival mobility are oftentimes coutcomes of uneven socioeconomic development, cf. Table~\ref{tab:1}.}
\begin{itemize}
\item{Livelihood and survival mobility are oftentimes coutcomes of uneven socioeconomic development.}
\item{Livelihood and survival mobility are oftentimes coutcomes of uneven socioeconomic development.}
\end{itemize}
\item{Livelihood and survival mobility are oftentimes coutcomes of uneven socioeconomic development.}
\end{itemize}

\begin{figure}[t]
\sidecaption[t]
% Use the relevant command for your figure-insertion program
% to insert the figure file.
% For example, with the option graphics use
\includegraphics[scale=.65]{figure}
%
% If no graphics program available, insert a blank space i.e. use
%\picplace{5cm}{2cm} % Give the correct figure height and width in cm
%
%\caption{Please write your figure caption here}
\caption{If the width of the figure is less than 7.8 cm use the \texttt{sidecapion} command to flush the caption on the left side of the page. If the figure is positioned at the top of the page, align the sidecaption with the top of the figure -- to achieve this you simply need to use the optional argument \texttt{[t]} with the \texttt{sidecaption} command}
\label{fig:2}       % Give a unique label
\end{figure}

\runinhead{Run-in Heading Boldface Version} Use the \LaTeX\ automatism for all your cross-references and citations as has already been described in Sect.~\ref{sec:2}.

\subruninhead{Run-in Heading Boldface and Italic Version} Use the \LaTeX\ automatism for all your cross-refer\-ences and citations as has already been described in Sect.~\ref{sec:2}\index{paragraph}.

\subsubruninhead{Run-in Heading Displayed Version} Use the \LaTeX\ automatism for all your cross-refer\-ences and citations as has already been described in Sect.~\ref{sec:2}\index{paragraph}.
% Use the \index{} command to code your index words
%
% For tables use
%
\begin{table}[!t]
\caption{Please write your table caption here}
\label{tab:1}       % Give a unique label
%
% Follow this input for your own table layout
%
\begin{tabular}{p{2cm}p{2.4cm}p{2cm}p{4.9cm}}
\hline\noalign{\smallskip}
Classes & Subclass & Length & Action Mechanism  \\
\noalign{\smallskip}\svhline\noalign{\smallskip}
Translation & mRNA$^a$  & 22 (19--25) & Translation repression, mRNA cleavage\\
Translation & mRNA cleavage & 21 & mRNA cleavage\\
Translation & mRNA  & 21--22 & mRNA cleavage\\
Translation & mRNA  & 24--26 & Histone and DNA Modification\\
\noalign{\smallskip}\hline\noalign{\smallskip}
\end{tabular}
$^a$ Table foot note (with superscript)
\end{table}
%
\section{Section Heading}
\label{sec:3}
% Always give a unique label
% and use \ref{<label>} for cross-references
% and \cite{<label>} for bibliographic references
% use \sectionmark{}
% to alter or adjust the section heading in the running head
Instead of simply listing headings of different levels we recommend to let every heading be followed by at least a short passage of text.  Further on please use the \LaTeX\ automatism for all your cross-references and citations as has already been described in Sect.~\ref{sec:2}.

Please note that the first line of text that follows a heading is not indented, whereas the first lines of all subsequent paragraphs are.

If you want to list definitions or the like we recommend to use the enhanced \verb|description| environment -- it will automatically rendered in line with the preferred layout.

\begin{description}[Type 1]
\item[Type 1]{That addresses central themes pertainng to migration, health, and disease. In Sect.~\ref{sec:1}, Wilson discusses the role of human migration in infectious disease distributions and patterns.}
\item[Type 2]{That addresses central themes pertainng to migration, health, and disease. In Sect.~\ref{subsec:2}, Wilson discusses the role of human migration in infectious disease distributions and patterns.}
\end{description}

\subsection{Subsection Heading} %
In order to avoid simply listing headings of different levels we recommend to let every heading be followed by at least a short passage of text. Use the \LaTeX\ automatism for all your cross-references and citations citations as has already been described in Sect.~\ref{sec:2}.

Please note that the first line of text that follows a heading is not indented, whereas the first lines of all subsequent paragraphs are.

\begin{svgraybox}
If you want to emphasize complete paragraphs of texts we recommend to use the newly defined class option \verb|graybox| and the newly defined environment \verb|svgraybox|. This will produce a 15 percent screened box 'behind' your text.

If you want to emphasize complete paragraphs of texts we recommend to use the newly defined class option and environment \verb|svgraybox|. This will produce a 15 percent screened box 'behind' your text.
\end{svgraybox}


\subsubsection{Subsubsection Heading}
Instead of simply listing headings of different levels we recommend to let every heading be followed by at least a short passage of text.  Further on please use the \LaTeX\ automatism for all your cross-references and citations as has already been described in Sect.~\ref{sec:2}.

Please note that the first line of text that follows a heading is not indented, whereas the first lines of all subsequent paragraphs are.

\begin{theorem}
Theorem text goes here.
\end{theorem}
%
% or
%
\begin{definition}
Definition text goes here.
\end{definition}

\begin{proof}
%\smartqed
Proof text goes here.
%\qed
\end{proof}

\paragraph{Paragraph Heading} %
Instead of simply listing headings of different levels we recommend to let every heading be followed by at least a short passage of text.  Further on please use the \LaTeX\ automatism for all your cross-references and citations as has already been described in Sect.~\ref{sec:2}.

Note that the first line of text that follows a heading is not indented, whereas the first lines of all subsequent paragraphs are.
%
% For built-in environments use
%
\begin{theorem}
Theorem text goes here.
\end{theorem}
%
\begin{definition}
Definition text goes here.
\end{definition}
%
\begin{proof}
%\smartqed
Proof text goes here.
%\qed
\end{proof}
%
\begin{trailer}{Trailer Head}
If you want to emphasize complete paragraphs of texts in an \verb|Trailer Head| we recommend to
use  \begin{verbatim}\begin{trailer}{Trailer Head}
...
\end{trailer}\end{verbatim}
\end{trailer}
%
\begin{question}{Questions}
If you want to emphasize complete paragraphs of texts in an \verb|Questions| we recommend to
use  \begin{verbatim}\begin{question}{Questions}
...
\end{question}\end{verbatim}
\end{question}
\eject%
\begin{important}{Important}
If you want to emphasize complete paragraphs of texts in an \verb|Important| we recommend to
use  \begin{verbatim}\begin{important}{Important}
...
\end{important}\end{verbatim}
\end{important}
%
\begin{warning}{Attention}
If you want to emphasize complete paragraphs of texts in an \verb|Attention| we recommend to
use  \begin{verbatim}\begin{warning}{Attention}
...
\end{warning}\end{verbatim}
\end{warning}

\begin{programcode}{Program Code}
If you want to emphasize complete paragraphs of texts in an \verb|Program Code| we recommend to
use

\verb|\begin{programcode}{Program Code}|

\verb|\begin{verbatim}...\end{verbatim}|

\verb|\end{programcode}|

\end{programcode}
%
\begin{tips}{Tips}
If you want to emphasize complete paragraphs of texts in an \verb|Tips| we recommend to
use  \begin{verbatim}\begin{tips}{Tips}
...
\end{tips}\end{verbatim}
\end{tips}
\eject
%
\begin{overview}{Overview}
If you want to emphasize complete paragraphs of texts in an \verb|Overview| we recommend to
use  \begin{verbatim}\begin{overview}{Overview}
...
\end{overview}\end{verbatim}
\end{overview}
\begin{backgroundinformation}{Background Information}
If you want to emphasize complete paragraphs of texts in an \verb|Background|
\verb|Information| we recommend to
use

\verb|\begin{backgroundinformation}{Background Information}|

\verb|...|

\verb|\end{backgroundinformation}|
\end{backgroundinformation}
\begin{legaltext}{Legal Text}
If you want to emphasize complete paragraphs of texts in an \verb|Legal Text| we recommend to
use  \begin{verbatim}\begin{legaltext}{Legal Text}
...
\end{legaltext}\end{verbatim}
\end{legaltext}
%
\begin{acknowledgement}
If you want to include acknowledgments of assistance and the like at the end of an individual chapter please use the \verb|acknowledgement| environment -- it will automatically be rendered in line with the preferred layout.
\end{acknowledgement}
%
\section*{Appendix}
\addcontentsline{toc}{section}{Appendix}
%
%
When placed at the end of a chapter or contribution (as opposed to at the end of the book), the numbering of tables, figures, and equations in the appendix section continues on from that in the main text. Hence please \textit{do not} use the \verb|appendix| command when writing an appendix at the end of your chapter or contribution. If there is only one the appendix is designated ``Appendix'', or ``Appendix 1'', or ``Appendix 2'', etc. if there is more than one.

\begin{equation}
a \times b = c
\end{equation}

\bibliographystyle{spmpsci.bst}
\bibliography{references.tex}
\input{references}
\end{document}
