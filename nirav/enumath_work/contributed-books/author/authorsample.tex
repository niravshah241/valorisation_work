%%%%%%%%%%%%%%%%%%%% author.tex %%%%%%%%%%%%%%%%%%%%%%%%%%%%%%%%%%%
%
% sample root file for your "contribution" to a contributed volume
%
% Use this file as a template for your own input.
%
%%%%%%%%%%%%%%%% Springer %%%%%%%%%%%%%%%%%%%%%%%%%%%%%%%%%%


% RECOMMENDED %%%%%%%%%%%%%%%%%%%%%%%%%%%%%%%%%%%%%%%%%%%%%%%%%%%
\documentclass[graybox]{svmult}

% choose options for [] as required from the list
% in the Reference Guide

\usepackage{type1cm}        % activate if the above 3 fonts are
                            % not available on your system
%
\usepackage{makeidx}         % allows index generation
\usepackage{graphicx}        % standard LaTeX graphics tool
                             % when including figure files
\usepackage{multicol}        % used for the two-column index
\usepackage[bottom]{footmisc}% places footnotes at page bottom


\usepackage{newtxtext}       % 
\usepackage{newtxmath}       % selects Times Roman as basic font

\usepackage{url}

%%Nirav added
\newenvironment{spmatrix}[1]
 {\def\mysubscript{#1}\mathop\bgroup\begin{pmatrix}}
 {\end{pmatrix}\egroup_{\textstyle\mathstrut\mysubscript}}
\DeclareMathOperator{\Tr}{Tr}
\DeclareMathOperator{\spn}{span}
\usepackage{bm}
\usepackage{tikz}
\usepackage{wrapfig}
\usepackage[export]{adjustbox}
\usepackage{subcaption}
\usepackage[numbers]{natbib}
\usepackage{float}
\captionsetup{compatibility=false}
%%Nirav added over

% see the list of further useful packages
% in the Reference Guide

\makeindex             % used for the subject index
                       % please use the style svind.ist with
                       % your makeindex program

%%%%%%%%%%%%%%%%%%%%%%%%%%%%%%%%%%%%%%%%%%%%%%%%%%%%%%%%%%%%%%%%%%%%%%%%%%%%%%%%%%%%%%%%%

\begin{document}

\title{Discontinuous Galerkin model order reduction of geometrically parametrized Stokes flow}
% Use \titlerunning{Short Title} for an abbreviated version of
% your contribution title if the original one is too long
\author{Nirav Vasant Shah, Martin Hess and Gianluigi Rozza}
% Use \authorrunning{Short Title} for an abbreviated version of
% your contribution title if the original one is too long
\institute{Nirav Vasant Shah \at Scuola Internazionale Superiore di Studi Avanzati - via Bonomea, 265 - 34136 Trieste ITALY, \email{snirav@sissa.it}
\and Martin Hess \at Scuola Internazionale Superiore di Studi Avanzati - via Bonomea, 265 - 34136 Trieste ITALY \email{martin.hess@sissa.it}}
\titlerunning{DG MOR of gemetrically parametrized Stokes flow}
\authorrunning{Shah et. al.}
%
% Use the package "url.sty" to avoid
% problems with special characters
% used in your e-mail or web address
%
\maketitle

\abstract*{The present work focuses on the geometrical parametrization and the reduced order modeling of Stokes flow. We discuss the concept of a parametrized geometry and its application within a reduced order modeling technique.  The full order model is based on discontinuous Galerkin method with an interior penalty formulation. We introduce broken Sobolev spaces, the jump and mean operator, as well as weak formulation require for an affine parameter dependency. The operators are transformed from a fixed domain to a parameter dependent domain using the affine parameter dependency. The proper orthogonal decomposition is used to obtain the basis of functions of the reduced order model. By using the Galerkin projection the linear system is projected onto the reduced space. During this process, offline-online decomposition is used to separate parameter dependent operation from parameter independent operations. Finally this technique is applied to an obstacle test problem. The numerical outcomes presented include the experimental error analysis and measurement of online simulation time.\\
\textbf{Keywords} Discontinuous Galerkin method, Stokes flow, Geometric parametrization, Proper orthogonal decomposition }

\abstract{The present work focuses on the geometrical parametrization and the reduced order modeling of the Stokes equation. We discuss the concept of a parametrized geometry and its application within a reduced order modeling technique.  The full order model is based on the discontinuous Galerkin method with an interior penalty formulation. We introduce the broken Sobolev spaces as well as the weak formulation required for an affine parameter dependency. The operators are transformed from a fixed domain to a parameter dependent domain using the affine parameter dependency. The proper orthogonal decomposition is used to obtain the basis of functions of the reduced order model. By using the Galerkin projection the linear system is projected onto the reduced space. During this process, offline-online decomposition is used to separate parameter dependent operation from parameter independent operations. Finally this technique is applied to an obstacle test problem. The numerical outcomes presented include experimental error analysis, eigenvalue decay and measurement of online simulation time.\\
\textbf{Keywords} Discontinuous Galerkin method, Stokes flow, Geometric parametrization, Proper orthogonal decomposition }

\section{Introduction}
\label{introduction}

The subject of the mathematical applications in fluid mechanics starts with one of the variants of the Navier-Stokes equation. In case of laminar flow, i.e. when fluctuations are negligible, this linearized form of the Navier-Stokes equation is the Stokes equation.

Discontinuous Galerkin Method (DGM) has found traction as numerical method for the elliptic problems ~\cite{peraire} as well as the hyperbolic problems ~\cite{hyperbolic}. DGM uses polynomial approximation of a suitable degree providing higher accuracy as well as allows discontinuity at the interface, by the concept of numerical flux, allowing greater flexibility. This fact makes DGM naturally attractive to problems, such as shock capturing, due to presence of steep gradients or discontinuities. Additionally, since the Dirichlet conditions are applied as boundary penalty, it avoids the necessity to construct a subspace of Sobolev space.

Geometric parametrization has emerged as a important application of the Parametric Partial Differential Equations (PPDEs) and as an alternative to the shape optimization. The concept of geometric parametrization allows to transfer operator evaluated on one geometric domain to another geometric domain efficiently. Model Order Reduction (MOR) on the other hand allows reducing the size of the system to be solved by working with the smaller system containing only dominant components. It is pertinent to mention that identifying the "dominant" components is critical to the success of the model order reduction strategy. The  faster computations obtained by MOR has helped in many query context, real time computation and quick transfer of computational results to industrial problems.

As evident from above advantages, the application of geometric parametrization and reduced order modeling to discontinuous Galerkin method will remain at the forefront of scientific work. The present work is aimed to contribute to this emerging field. 
%We first introduce the notion of geometric parametrization. We subsequently introduce Discontinuous Galerkin Interior Penalty Method (DG-IPM) for the stokes flow. We then explain using the affine parameter dependence and its application in the context of the offline-online decomposition. Proper Orthogonal Decomposition (POD) is applied to construct the reduced basis space and Galerkin projection is applied to project the system of equations on the reduced basis space. Finally we present an obstacle test problem, to which the introduced method is applied. The numerical outcomes provide overview of the potential of the presented method.

\section{Geometric parametrization}\label{geometric_parametrization_section}

Consider $\Omega = \Omega(\mu) \in \mathbb{R}^d$ as an open bounded domain. The parameter set $\mu \in \mathbb{P}$, where $\mathbb{P}$ is parameter space, completely characterizes the domain. Also, consider a parameter set $\bar{\mu} \in \mathbb{P}$, as the known parameter set and $\Omega(\bar{\mu})$ as the reference domain, whose configuration is completely known. The invertible mapping $\bm{F}(\cdot,\mu) : \Omega(\bar{\mu}) \rightarrow \Omega(\mu)$ links the reference domain and the parametrized domain. In the case of affine transformation, $\bm{F}$ is of the form,
\begin{equation}\label{affine_F}
\begin{split}
x = \bm{F}(\hat{x},\mu) = \bm{G}_F(\mu)\hat{x} + c_F(\mu) \ ; \forall x \in \Omega \ , \ \hat{x} \in \Omega(\bar{\mu}) \ , \\ \bm{G}_F(\mu) \in \mathbb{R}^{d \times d} \ , \ c_F \in \mathbb{R}^{d \times 1} \ .
\end{split}
\end{equation}
The boundary of $\Omega(\mu)$, that is $\partial \Omega(\mu)$ is divided into Neumann boundary $\Gamma_N(\mu)$ and Dirichlet boundary $\Gamma_D(\mu)$ i.e. $\partial \Omega(\mu) = \Gamma_N(\mu) \cup \Gamma_D(\mu)$. In order to have $\bm{F}(\hat{x},\mu)$ affine form, the domain $\Omega(\mu)$ is divided into $n_{su}$ triangular subdomains such that $\Omega(\mu) = \bigcup\limits_{i=1}^{n_{su}} \Omega_i(\mu) \ , \ \Omega_i(\mu) \bigcap \Omega_j(\mu) = \emptyset \ , \ \text{for} \ i \neq j$.

\section{Discontinuous Galerkin formulation}
\label{DG_formulation}

The domain $\Omega$ is divided into $N_{el}$ number of triangular elements $\tau_k$ such that $\Omega = \bigcup\limits_{k=1}^{N_{el}} \tau_k$. The triangulation $\mathcal{T}$ is the set of all triangular elements i.e. $\mathcal{T} = \lbrace \tau_k \rbrace_{k=1}^{N_{el}}$. The internal boundary is denoted by $\Gamma = \bigcup\limits_{k=1}^{N_{el}} \partial \tau_k \backslash \partial \Omega$. The vertices of triangles $\lbrace \tau_k \rbrace_{k=1}^{N_{el}}$ are called nodes. $\overrightarrow{n}$ is the outward pointing normal to an edge of element.

The Stokes's equation in strong form can be stated as,
\begin{flalign}\label{stokes_strong_form}
\begin{split}
\text{Stokes equation: } & -\nu \Delta \overrightarrow{u} + \nabla p = \overrightarrow{f} \ , \ \text{in } \Omega \ , \\
\text{Continuity equation: } & \nabla \cdot \overrightarrow{u} = 0 \ , \ \text{in} \ \Omega \ , \\
\text{Dirichlet condition: } & \overrightarrow{u} = \overrightarrow{u}_D \ , \ \text{on } \Gamma_D \ , \\
\text{Neumann condition: } & -p \overrightarrow{n} + \nu \overrightarrow{n} \cdot \nabla \overrightarrow{u} = \overrightarrow{t} \ , \ \text{on} \ \Gamma_N \ .
\end{split}
\end{flalign}

The velocity vector field $\overrightarrow{u}$ and pressure scalar field $p$ are the unknowns. $\nu$ is the material property known as kinematic viscosity. Vector $\overrightarrow{f}$ is the external force term or source term. $\overrightarrow{u}_D$ is the Dirichlet velocity and vector $\overrightarrow{t}$ is the Neumann value.

Before introducing the weak form let us introduce the broken Sobolev spaces for variables.

\begin{equation*} \label{velocity_pressure_test}
\begin{split}
\text{For velocity: } \mathbb{V} = \lbrace \overrightarrow{\phi} \in (L^2(\Omega))^d | \ \overrightarrow{\phi} |_{\tau_k} \in (P^D(\tau_k))^d \ , \ \tau_k \in \mathcal{T} \rbrace \ , \\
\text{For pressure: } \mathbb{Q} = \lbrace \psi \in (L^2(\Omega)) | \ \psi |_{\tau_k} \in (P^{D-1}(\tau_k)) \ , \ \tau_k \in \mathcal{T} \rbrace \ .
\end{split}
\end{equation*}
Here, $P^D(\tau_k)$ denotes space of polynomials of degree at most $D, \ D \geq 2$ over $\tau_k$.

In finite dimensional or discrete system, velocity approximation $\overrightarrow{u}_h(x)$ and pressure approximation $p_h(x)$ at any point $x \in \Omega$ are given by,
\begin{equation}\label{velocity_pressure_coefficients}
\overrightarrow{u}_h(x) = \sum\limits_{i=1}^{u_{ndofs}} \overrightarrow{\phi}_i \hat{u}_i \ , \
p_h(x) = \sum\limits_{i=1}^{p_{ndofs}} \psi_i \hat{p}_i \ ,
\end{equation}
where $\hat{u}_i$'s and $\hat{p}_i$'s are coefficients of velocity basis functions and pressure basis functions respectively. 

We expect that $\overrightarrow{u}_h \rightarrow \overrightarrow{u}$ and $p_h \rightarrow p$ as $u_{ndofs} \rightarrow \infty$ and $p_{ndofs} \rightarrow \infty$ respectively. Considering scope of present work, the convergence analysis will not be discussed here. The readers are advised to refer to \cite{pacciarini}, \cite{riviere}. 

%\begin{figure}
%\centering
%\sidecaption[t]
%\begin{tikzpicture}
%\draw (0,0) node[anchor=north]{${}$}
%  -- (4,0) node[anchor=north]{${}$}
%  -- (0,4) node[anchor=south]{${}$}
%  -- cycle;
%\draw (4,4) node[anchor=north]{${}$}
%  -- (4,0) node[anchor=north]{${}$}
%  -- (0,4) node[anchor=south]{${}$}
%  -- cycle;
%\draw[->] (2,2)--(2.5,2.5)node[label={[xshift=0.4cm, yshift=-0.7cm]${\overrightarrow{n}^-}$}]{};
%\draw[->] (2,2)--(1.5,1.5)node[label={[xshift=0cm, yshift=0cm]${\overrightarrow{n}^+}$}]{};
%\node at (1,1){$\tau_{k}^-$};
%\node at (3,3){$\tau_{k}^+$};
%\node at (1.7,3){$\overrightarrow{\phi}^+,\psi^+$};
%\node at (2.3,1){$\overrightarrow{\phi}^-,\psi^-$};
%\end{tikzpicture}
%\caption{Definition of jump and mean operator. \newline The superscript $+$ refers to quantity in the element itself and the superscript $-$ refers to quantity in the neighboring element}
%\label{fig:Self_neighbour}
%\end{figure}

In the subsequent sections, $\left( \cdot \right),\left( \cdot \right)_{\Gamma_D},\left( \cdot \right)_{\Gamma_N},\left( \cdot \right)_{\Gamma}$ represent the $L^2$ scalar product over $\Omega,\Gamma_D,\Gamma_N,\Gamma$ respectively. The jump operator $\left[ \cdot \right]$ and the average operator $\lbrace \cdot \rbrace$ are important concepts in DGM formulation and are required to approximate the numerical flux. Different definitions are used in literatures for jump and mean operators. We use the jump and average operators as defined in \cite{jump_mean_operator}.

%The presence of normal vector $\overrightarrow{n}$ in jump and average operator introduced below allows symmetric formulation and also ensures that jump of a vector is vector and jump of a scalar is scalar.
%
%\begin{itemize}
%\item For vector quantity $\overrightarrow{\phi}$:
%\begin{itemize}
%\item Jump operator: 
%$\left[\overrightarrow{\phi} \cdot \overrightarrow{n}\right] = \overrightarrow{\phi}^+ \cdot \overrightarrow{n}^+ + \overrightarrow{\phi}^- \cdot \overrightarrow{n}^-$ on $\Gamma$, $\left[\overrightarrow{\phi} \cdot \overrightarrow{n}\right] = \overrightarrow{\phi} \cdot \overrightarrow{n}$ on $\partial \Omega$.
%\item Average operator:
%$\lbrace \overrightarrow{\phi} \rbrace = \frac{\overrightarrow{\phi}^+ + \overrightarrow{\phi}^-}{2}$ on $\Gamma$, $\lbrace \overrightarrow{\phi} \rbrace = \overrightarrow{\phi}$ on $\partial \Omega$.
%\end{itemize}
%\item For scalar quantity $\psi$:
%\begin{itemize}
%\item Jump operator:
%$\left[\psi \overrightarrow{n} \right] = \psi^+ \overrightarrow{n}^+ + \psi^- \overrightarrow{n}^-$ on $\Gamma$, $\left[\psi \overrightarrow{n} \right] = \psi \overrightarrow{n}$ on $\partial \Omega$.
%\item Average operator:
%$\lbrace \psi \rbrace = \frac{\psi^+ + \psi^-}{2}$ on $\Gamma$, $\lbrace \psi \rbrace = \psi$ on $\partial \Omega$. 
%\end{itemize}
%\end{itemize}

The weak form of the Stokes equation is as follow,
\begin{gather}\label{stokes_weak_ch3}
a_{IP}(\overrightarrow{u},\overrightarrow{\phi}) + b(\overrightarrow{\phi},p) + \left( \lbrace p \rbrace,[\overrightarrow{n} \cdot \overrightarrow{\phi}] \right)_{\Gamma \cup \Gamma_D} = l_{IP}(\overrightarrow{\phi}) \ ,
\end{gather}
\begin{equation}
\begin{split}
a_{IP}(\overrightarrow{u},\overrightarrow{\phi}) = \left( \nabla \overrightarrow{u}, \nabla \overrightarrow{\phi} \right) + C_{11} \left( [\overrightarrow{u}],[\overrightarrow{\phi}] \right)_{\Gamma \cup \Gamma_D} \\ - \nu \left( \lbrace \nabla \overrightarrow{u}\rbrace ,[\overrightarrow{n} \otimes \overrightarrow{\phi}] \right)_{\Gamma \cup \Gamma_D} - \nu \left( [\overrightarrow{n} \otimes \overrightarrow{u}], \lbrace \nabla \overrightarrow{\phi} \rbrace \right)_{\Gamma \cup \Gamma_D} \ ,
\end{split}
\end{equation}
\begin{gather}
b(\overrightarrow{\phi},\psi) = -\int_{\Omega} \psi \nabla \cdot \overrightarrow{\phi} \ , \\
l_{IP}(\overrightarrow{\phi}) = \left( \overrightarrow{f},\overrightarrow{\phi} \right) + \left( \overrightarrow{t},\overrightarrow{\phi} \right)_{\Gamma_N} + C_{11} \left(\overrightarrow{u}_D,\overrightarrow{\phi}\right)_{\Gamma_D} - \left( \overrightarrow{n} \otimes \overrightarrow{u}_D, \nu \nabla \overrightarrow{\phi} \right)_{\Gamma_D} \ .
\end{gather}

The penalty parameter $C_{11}>0$ is an empirical constant to be kept large enough to maintain coercivity of $a_{IP}(\overrightarrow{u},\overrightarrow{\phi})$.

The weak form of the continuity equation is as follow,
\begin{equation}\label{contiuity_weak_ch3}
\begin{split}
b(\overrightarrow{u},\psi) + ({\psi},[\overrightarrow{n} \cdot \overrightarrow{u}])_{\Gamma \cup \Gamma_D} = (\psi,\overrightarrow{n} \cdot \overrightarrow{u}_D)_{\Gamma_D} \ .
\end{split}
\end{equation}

In discrete form the system of equations can be written as, 
\begin{equation} \label{Stokes_matrix_ch3}
\begin{spmatrix}{\textrm{Stiffness matrix}}
    \bm{A} & \bm{B} \\
    \bm{B}^T & 0
\end{spmatrix}
\begin{spmatrix}{\textrm{Solution vector}}
    U \\
    P
\end{spmatrix}
=
\begin{spmatrix}{\textrm{Right hand side (Known)}}
    F_1  \\
    F_2
\end{spmatrix}
\textrm{.}
\end{equation}

Here, $\bm{A}_{ij} = a_{IP} (\overrightarrow{\phi}_i,\overrightarrow{\phi}_j)$, $\bm{B}_{ij} = b(\overrightarrow{\phi}_i,\psi_j) + \left( \lbrace \psi_j \rbrace , [n \cdot \overrightarrow{\phi}_i]\right)_{\Gamma \cup \Gamma_D}$, $F_1 = l_{IP}(\overrightarrow{\phi}_i)$ and $F_2 = \left( \psi_j,\overrightarrow{n} \cdot \overrightarrow{u}_D \right)_{\Gamma_D}$ for $i=1,\ldots,u_{ndofs}$ and $j=1,\ldots,p_{ndofs}$. The column vectors $U$ and $P$ are coefficients $\hat{u}$'s and $\hat{p}$'s from equation \eqref{velocity_pressure_coefficients}.

\section{Affine expansion}

We evaluate and solve the Stokes equation weak formulation on a reference domain $\Omega({\bar{\mu}})$. Given a parameter $\mu \neq \bar{\mu}$ we need to evaluate the linear systems of equation \eqref{Stokes_matrix_ch3} on new domain $\Omega(\mu)$. To accomplish this we use affine expansion using linear nature of equation and diving $\Omega(\bar{\mu})$ into triangular subdomains $\Omega_i(\bar{\mu}) \ , \ i = \lbrace 1,2,\ldots,n_{su} \rbrace$ as explained earlier in the section geometric parametrization [Section \ref{geometric_parametrization_section}]. The affine expansion of operators is essentially change of variable and has been explained in literatures such as \cite{CRBM}. However it is pertinent to explain two expansions as specific to DGM formulation.

\begin{itemize}
\item In order to transfer the terms containing jump and average operator following approach is used in present analysis.
\begin{equation*}\label{jump_average_term_split}
\begin{split}
\left(\lbrace \nabla \overrightarrow{\phi} \rbrace , \left[ \overrightarrow{n} \otimes \overrightarrow{\phi}  \right]  \right) = \left( \nabla \overrightarrow{\phi}^+ , \overrightarrow{n}^+ \otimes \overrightarrow{\phi}^+ \right) + \left( \nabla \overrightarrow{\phi}^+ , \overrightarrow{n}^- \otimes \overrightarrow{\phi}^- \right) + \\ 
\left( \nabla \overrightarrow{\phi}^- , \overrightarrow{n}^+ \otimes \overrightarrow{\phi}^+ \right) + \left( \nabla \overrightarrow{\phi}^- , \overrightarrow{n}^- \otimes \overrightarrow{\phi}^- \right) \ .
\end{split}
\end{equation*}
Each term on the right hand side of the above equation can now be transformed using the affine map.

\item The coercivity term $C_{11}\left( [\overrightarrow{\phi}],[\overrightarrow{u}] \right)_{\Gamma \cup \Gamma_D}$ is not transformed but used as evaluated on reference domain $\Omega(\bar{\mu})$. The affine transformation is given by,
\begin{equation*}
\begin{split}
C_{11}\left( [\overrightarrow{\phi}(\hat{x}),\overrightarrow{u}(\hat{x})] \right)_{\Gamma(\mu) \cup \Gamma_D(\mu)} = C_{11} \alpha \left( [\overrightarrow{\phi}(\bm{F}(\hat{x})),\overrightarrow{u}(\bm{F}(\hat{x}))] \right)_{\Gamma(\bar{\mu}) \cup \Gamma_D(\bar{\mu})} \ , \\
\alpha = \frac{\text{length of }\left( \Gamma(\mu) \cup \Gamma_D(\mu)\right)}{\text{length of }\left( \Gamma(\bar{\mu}) \cup \Gamma_D(\bar{\mu})\right)} \ , \ \hat{x} \in \Omega(\bar{\mu}) \ , \ x \in \Omega(\mu) \ .
\end{split}
\end{equation*}

Since, $C_{11}$ is empirical coefficient replacing $C_{11} \alpha$ with $C_{11}$ will not change the formulation as long as coercivity of $a_{IP}$ over parameter space $\mathbb{P}$ is maintained. 
%In the present analysis, $C_{11}$ is not calculated exactly but only order of magnitude required for $C_{11}$ is estimated and $C_{11}$ is kept one magnitude larger as safeguard against round-off errors. As long as the domain $\Omega(\mu)$ is not deformed much compared to its reference configuration $\Omega(\bar{\mu})$, $C_{11}$ and $C_{11}\alpha$ will have the same order of magnitude. However, regardless of this fact, large deformations are not favorable as it will lead to bad mesh quality and in turn, will lead to poor DGM-approximation.
\end{itemize}

\section{Reduced basis method}\label{rb_section}

In this section, the snapshot proper orthogonal decomposition method and the offline-online decomposition are briefly described. For detailed explanation, we refer to ~\cite{CRBM}.
%\subsection{Snapshot proper orthogonal decomposition}\label{POD_section}
In first step, the solutions based on $\mu_n, n \in \lbrace 1,....,n_s \rbrace$ are calculated i.e. $n_s$ snapshots are generated. The velocity snapshots and the pressure snapshots are stored in $\bm{S}_v \in \mathbb{R}^{u_{ndofs} \times n_s}$ and $\bm{S}_p \in \mathbb{R}^{p_{ndofs} \times n_s}$ respectively. Let us also introduce inner product matrices $\bm{M}_v \in \mathbb{R}^{u_{ndofs} \times u_{ndofs}}$ and $\bm{M}_p \in \mathbb{R}^{p_{ndofs} \times p_{ndofs}}$.

\begin{gather*}
\bm{M}_{v,ij} = \int_{\Omega} \overrightarrow{\phi}_i \cdot \overrightarrow{\phi}_j + \sum_{k=1}^{N_{el}} \int_{\tau_k} \nabla \overrightarrow{\phi}_i : \nabla \overrightarrow{\phi}_j \ , \ i,j = 1, \ldots, u_{ndofs} \ , \\
\bm{M}_{p,ij} = \int_{\Omega} \psi_i \psi_j \ , \ i,j = 1, \ldots, p_{ndofs} \ .
\end{gather*}

The dimension of reduced basis is denoted as $N$ and it is asserted that $N << u_{ndofs}, \ N < n_s$. Next, the spectral decomposition of the snapshots is performed.
\begin{equation}\label{snapshot_eigen_value}
\bm{S}_v^T \bm{M}_v \bm{S}_v = \bm{V} \bm{\Theta} \bm{V}^T \ .
\end{equation}
The columns of $\bm{V}$ are eigenvectors and $\Theta$ has eigenvalues $\theta_i \ , \ 1 \leq i,j \leq n_s$ in sorted order ($\theta_1 \geq \ldots \geq \theta_{n_s}$) such that, $\Theta_{ij} = \theta_i \delta_{ij}$.

%The reduced basis space for velocity is formed by the linear combination of the snapshot vector,
%\begin{equation}\label{linear_combination_snapshots}
%\bm{B}_v = \bm{S}_v \bm{A} \ , \ \bm{A} \in \mathbb{R}^{n_s \times N} \ .
%\end{equation}

The reduced basis for the velocity $\overrightarrow{\phi}_N$ is formed by,
\begin{equation}
\overrightarrow{\phi}_N = \overrightarrow{\phi}^T \bm{B}_v \ , \\
\bm{B}_v = \bm{S}_v \bm{V} \bm{\Theta}^{-\frac{1}{2}} \bm{R} \ , \ \bm{R} = [\bm{I}_{u_{ndofs} \times N} ; \bm{0}_{(n_s-N) \times N}] \ .
\end{equation}
%Considering orthonormality of the reduced basis $\overrightarrow{\phi}_N$ with respect to inner product given by $\bm{M}_v$, i.e. $\bm{B}_v^T \bm{M}_v \bm{B}_v = \bm{I}_{n_s \times n_s}$,
%\begin{equation}
%\bm{B}_v = \bm{S}_v \bm{V} \bm{\Theta}^{-\frac{1}{2}} \bm{R} \ , \ \bm{R} = [\bm{I}_{u_{ndofs} \times N} ; \bm{0}_{(n_s-N) \times N}] \ ,
%\end{equation}
where, $\bm{I}_{N \times N}$ is identity matrix of size $N \times N$.
The reduced basis space $\bm{B}_p$ can be generated in similar manner using the pressure snapshots $\bm{S}_p$ and the inner product matrix $\bm{M}_p$. Above procedure is performed during offline phase.

The discrete system of equations is projected onto the reduced basis space by Galerkin projection as,
\begin{equation} \label{Stokes_matrix_reduced}
\begin{spmatrix}{\tilde{K}}
    \bm{B}_v^T \bm{A}(\mu) \bm{B}_v & \bm{B}_v^T \bm{B}(\mu) \bm{B}_p \\
    \bm{B}_p^T \bm{B}(\mu)^T \bm{B}_v & \bm{0}
\end{spmatrix}
\begin{spmatrix}{\zeta}
    U_N \\
    P_N
\end{spmatrix}
=
\begin{spmatrix}{\tilde{F}}
    \bm{B}_v^T F_1(\mu)  \\
    \bm{B}_p^T F_2(\mu)
\end{spmatrix} \ .
\end{equation}
The solution vectors $U$ and $P$ (equation \eqref{Stokes_matrix_ch3}) are then computed as, $U = \bm{B}_v U_N \ , \ P = \bm{B}_p P_N$.

The projection onto reduced basis, solution of smaller system of equations and computation of $U$ and $P$ are steps performed during online phase.

%\subsection{Galerkin reduced basis formulation}\label{Galerkin_section}
%
%The discrete system of equations is projected onto the reduced basis space as,
%\begin{equation} \label{Stokes_matrix_reduced}
%\begin{spmatrix}{\tilde{K}}
%    \bm{B}_v^T \bm{A}(\mu) \bm{B}_v & \bm{B}_v^T \bm{B}(\mu) \bm{B}_p \\
%    \bm{B}_p^T \bm{B}(\mu)^T \bm{B}_v & \bm{0}
%\end{spmatrix}
%\begin{spmatrix}{\zeta}
%    U_N \\
%    P_N
%\end{spmatrix}
%=
%\begin{spmatrix}{\tilde{F}}
%    \bm{B}_v^T F_1(\mu)  \\
%    \bm{B}_p^T F_2(\mu)
%\end{spmatrix} \ ,
%\end{equation}
%and accordingly the variational form for the reduced degrees of freedom $\zeta$ is solved,
%\begin{equation}
%\tilde{\bm{K}} \zeta = \tilde{F} \ .
%\end{equation}
%The parameter dependent matrices in equation \eqref{Stokes_matrix_reduced} are evaluated by using the affine decomposition.
%The solution vectors $U$ and $P$ (equation \eqref{Stokes_matrix_ch3}) are then computed as,
%\begin{equation}
%U = \bm{B}_v U_N \ , \ P = \bm{B}_p P_N \ .
%\end{equation}
%The procedure explained in this section is performed during online phase.

%\subsection{Offline-online procedure}
%
%During the offline phase $n_s$ snapshots are computed and reduced basis spaces $\bm{B}_v$ and $\bm{B}_p$ are created as outlined in section \ref{POD_section}. During the online phase the systems of equations are projected on reduced space using Galerkin projection. During the online phase, the smaller systems of equation obtained by Galerkin projection is solved and the reduced basis solution is computed as outlined in section \ref{Galerkin_section}. 

\section{Numerical example}

We perform the POD-Galerkin method as mentioned in the section \ref{rb_section}. The numerical experiments were performed using RBmatlab ~\cite{rbmatlab},~\cite{master_thesis}. The reference domain $\Omega({\bar{\mu}})$ is the unit square domain $[0,1] \times [0,1]$ with triangle with vertices $(0.3,0),(0.5,0.3),(0.7,0)$ as obstacle. The geometric parameters are the coordinates of the tip of the obstacle i.e. $\bar{\mu} = (0.5,0.3)$. The boundary ${x=0}$ is Dirichlet boundary with inflow velocity at point $(0,y)$ as $u = (y(1-y), 0)$. The boundary ${x = 1}$ is a Neumann boundary with zero Neumann value i.e. $\overrightarrow{t} = (0, 0)$. Other boundaries are Dirichlet boundary with no slip condition. The source term is $\overrightarrow{f} = (0,0)$.

The training set was generated by random generation of $100$ parameters within the interval $[0.4,0.6] \times [0.4,0.6]$. The test set contained $10$ random parameters within the interval $[0.4,0.6] \times [0.4,0.6]$. For velocity basis function polynomial of degree $P^D = 2$ and for pressure basis function polynomial of degree $P^{D-1} = 1$ was used. The number of velocity degrees of freedom and pressure degrees of freedom were $u_{ndofs} = 4704$ and $p_{ndofs} = 1176$ respectively.

Figure \ref{dg_rb_solution_47_33} shows solution computed by DGM and Reduced Basis (RB) at parameter value $\mu = (0.47,33)$. Figure \ref{error_vs_basis} shows error vs size of reduced basis space. The drop in error w.r.t to increased size of reduced basis is inline with the expectation based on eigenvalue decay (Figure \ref{ev_decay}). The online simulation time for reduced basis with $10$ basis functions was $2.25$ seconds resulting in speedup of $19.5$ as compared to full order model.

\begin{figure}[H] %[t!] % "[t!]" placement specifier just for this example
\begin{subfigure}{0.31\textwidth}
\includegraphics[width=\linewidth]{offline_velocity_1_at_47_33.jpg}
\caption{Velocity $x-$direction DG solution} \label{vel_x_dg}
\end{subfigure}\hspace*{\fill}
\begin{subfigure}{0.31\textwidth}
\includegraphics[width=\linewidth]{online_velocity_1_at_47_33.jpg}
\caption{Velocity $x-$direction RB solution} \label{vel_x_rb}
\end{subfigure}
\begin{subfigure}{0.31\textwidth}
\includegraphics[width=\linewidth]{velocity_error_1_at_47_33.jpg}
\caption{$x-$component of Velocity absolute error $\overrightarrow{u}_h-\overrightarrow{u}_N$} \label{error_x_vel}
\end{subfigure}

\begin{subfigure}{0.31\textwidth}
\includegraphics[width=\linewidth]{offline_velocity_2_at_47_33.jpg}
\caption{Velocity $y-$direction DG solution} \label{vel_y_dg}
\end{subfigure}\hspace*{\fill}
\begin{subfigure}{0.31\textwidth}
\includegraphics[width=\linewidth]{online_velocity_2_at_47_33.jpg}
\caption{Velocity $y-$direction RB solution} \label{vel_y_rb}
\end{subfigure}
\begin{subfigure}{0.31\textwidth}
\includegraphics[width=\linewidth]{velocity_error_2_at_47_33.jpg}
\caption{$y-$component of Velocity absolute error $\overrightarrow{u}_h-\overrightarrow{u}_N$} \label{error_y_vel}
\end{subfigure}

\begin{subfigure}{0.31\textwidth}
\includegraphics[width=\linewidth]{offline_pressure_at_47_33.jpg}
\caption{Pressure DG solution} \label{pre_dg}
\end{subfigure}\hspace*{\fill}
\begin{subfigure}{0.31\textwidth}
\includegraphics[width=\linewidth]{online_pressure_at_47_33.jpg}
\caption{Pressure RB solution} \label{pre_rb}
\end{subfigure}
\begin{subfigure}{0.31\textwidth}
\includegraphics[width=\linewidth]{pressure_error_at_47_33.jpg}
\caption{Pressure absolute error $p_h-p_N$} \label{pre_error}
\end{subfigure}
\caption{DG and RB solution $\mu = [\mu_x \ \mu_y] = [0.47 \ 0.33]$} 
\label{dg_rb_solution_47_33}
\end{figure}

\begin{figure}[H]
\begin{subfigure}{0.48\textwidth}
\includegraphics[width=\linewidth]{size_vs_maximum_reduced_basis_velocity_error_semilog.jpg}
\caption{Size of reduced basis space vs. Maximum relative error in velocity with inner product induced by $\bm{M}_v$} \label{error_vs_basis_velocity}
\end{subfigure}\hspace*{\fill}
\begin{subfigure}{0.48\textwidth}
\includegraphics[width=\linewidth]{size_vs_maximum_reduced_basis_pressure_error_semilog.jpg}
\caption{Size of reduced basis space vs. Maximum relative error in pressure with inner product induced by $\bm{M}_p$} \label{error_vs_basis_pressure}
\end{subfigure}
  \caption{Size of reduced basis vs Maximum relative error} 
\label{error_vs_basis}
\end{figure}

\begin{figure}[H]
\begin{subfigure}{0.31\textwidth}
\includegraphics[width=\linewidth]{x_velocity_eigen_value_semilog.jpg}
\caption{$x-$Velocity eigenvalues (semilog scale)} \label{vel_x_ev}
\end{subfigure}\hspace*{\fill}
\begin{subfigure}{0.31\textwidth}
\includegraphics[width=\linewidth]{y_velocity_eigen_value_semilog.jpg}
\caption{$y-$Velocity eigenvalues (semilog scale)} \label{vel_y_ev}
\end{subfigure}
\begin{subfigure}{0.31\textwidth}
\includegraphics[width=\linewidth]{pressure_eigen_value_semilog.jpg}
\caption{Pressure eigenvalues (semilog scale)} \label{pressure_ev}
\end{subfigure}
\caption{Eigenvalue decay}\label{ev_decay}
\end{figure}

%\section{Acknowledgements}
%
%\begin{itemize}
%
%\item The numerical experiments were performed using RBmatlab (\url{https://www.morepas.org/software/rbmatlab/0.11.04/doc/index.html}).
%
%\item This activity had received partial funding from below two grants.
%\begin{itemize}
%\item AROMA-CFD, H2020 ERC CoG 2015 GA 681447 by European Research Council (ERC)
%\item COST Action TD1307 by European Model Reduction Network (EU-MORNET)
%\end{itemize}
%The author is thankful to the organizers for supporting training, research and networking activity.
%
%\end{itemize}

\bibliographystyle{spbasic}
\bibliography{references}
%%%%%%%%%%%%%%%%%%%%% author.tex %%%%%%%%%%%%%%%%%%%%%%%%%%%%%%%%%%%
%
% sample root file for your "contribution" to a contributed volume
%
% Use this file as a template for your own input.
%
%%%%%%%%%%%%%%%% Springer %%%%%%%%%%%%%%%%%%%%%%%%%%%%%%%%%%


% RECOMMENDED %%%%%%%%%%%%%%%%%%%%%%%%%%%%%%%%%%%%%%%%%%%%%%%%%%%
\documentclass[graybox]{svmult}

% choose options for [] as required from the list
% in the Reference Guide

\usepackage{type1cm}        % activate if the above 3 fonts are
                            % not available on your system
%
\usepackage{makeidx}         % allows index generation
\usepackage{graphicx}        % standard LaTeX graphics tool
                             % when including figure files
\usepackage{multicol}        % used for the two-column index
\usepackage[bottom]{footmisc}% places footnotes at page bottom


\usepackage{newtxtext}       % 
\usepackage{newtxmath}       % selects Times Roman as basic font

\usepackage{url}

%%Nirav added
\newenvironment{spmatrix}[1]
 {\def\mysubscript{#1}\mathop\bgroup\begin{pmatrix}}
 {\end{pmatrix}\egroup_{\textstyle\mathstrut\mysubscript}}
\DeclareMathOperator{\Tr}{Tr}
\DeclareMathOperator{\spn}{span}
%%Nirav added over

% see the list of further useful packages
% in the Reference Guide

\makeindex             % used for the subject index
                       % please use the style svind.ist with
                       % your makeindex program

%%%%%%%%%%%%%%%%%%%%%%%%%%%%%%%%%%%%%%%%%%%%%%%%%%%%%%%%%%%%%%%%%%%%%%%%%%%%%%%%%%%%%%%%%

\begin{document}

\title*{Reduced order modeling of geometrically parametrized discontinuous Galerkin formulation for the Stokes equation}
% Use \titlerunning{Short Title} for an abbreviated version of
% your contribution title if the original one is too long
\author{Nirav Vasant Shah, Martin Hess and Gianluigi Rozza}
% Use \authorrunning{Short Title} for an abbreviated version of
% your contribution title if the original one is too long
\institute{Nirav Vasant Shah \at Scuola Internazionale Superiore di Studi Avanzati - via Bonomea, 265 - 34136 Trieste ITALY, \email{snirav@sissa.it}
\and Martin Hess \at Scuola Internazionale Superiore di Studi Avanzati - via Bonomea, 265 - 34136 Trieste ITALY \email{martin.hess@sissa.it}}
%
% Use the package "url.sty" to avoid
% problems with special characters
% used in your e-mail or web address
%
\maketitle

\abstract*{Each chapter should be preceded by an abstract (no more than 200 words) that summarizes the content. The abstract will appear \textit{online} at \url{www.SpringerLink.com} and be available with unrestricted access. This allows unregistered users to read the abstract as a teaser for the complete chapter.
Please use the 'starred' version of the \texttt{abstract} command for typesetting the text of the online abstracts (cf. source file of this chapter template \texttt{abstract}) and include them with the source files of your manuscript. Use the plain \texttt{abstract} command if the abstract is also to appear in the printed version of the book.}

\abstract{The present work focuses on geometrical parametrization and reduced order modeling of Stokes flow. The importance of Stokes flow, advantages of discontinuous Galerkin method are introduced first. We also discuss the concept of geometric parametrization and its application along with importance of reduced order model technique.  The full order model is based on discontinuous Galerkin method interior penalty formulation. The concepts of broken Sobolev spaces, relevant norms, jump and mean operator are introduced. The weak formulation is derived based in suitable space to obtain the full order model. We then introduce the concept of geometric parametrization. The operators are transformed from fixed domain to parameter dependent domain by exploring affine parameter dependence which results in efficient assembly of system matrix. Thereafter, proper orthogonal decomposition is applied to obtain basis for function space for reduced order model. By using Galerkin projection the linear system to be solved is projected onto reduced space. During the process, offline-online decomposition is used to separate computation of expensive parameter independent part and fast parameter independent part. Finally the technique is applied to test problem. The numerical outcomes presented include the experimental error analysis, eigenvalue computation and measurement of online simulation time. \cite{psysoc-journal}}

\section{Introduction}
\label{introduction}

The subject of mathematical applications in fluid mechanics starts with one of the variants of the Navier-Stokes equations, such as the Stokes equation. Almost all processes of fluid mechanics require considerations related to the Navier-Stokes equations. Navier-Stokes equation is non-linear, characterizing flow fluctuations. However, in case of laminar flow, i.e. when fluctuations are negligible, this linearized form of the Navier-Stokes equation is the Stokes equation.

Discontinuous Galerkin method (DGM) has found traction as numerical method for elliptic problems \textbf{pereire reference} as well as hyperbolic problems \textbf{Book on compressible flow reference}. This is due to its several advantages over Finite Element Method (FEM) and Finite Volume Method (FVM). In fact, DG method is considered as combination of FEM and FVM. DGM uses polynomial approximation of suitable degree providing higher accuracy as well as allows discontinuity at the interface, by the concept of numerical flux, allowing greater flexibility. This fact makes DGM naturally attractive to problems such as shock capturing due to presence of steep gradients or discontinuities. Additionally, since the Dirichlet conditions are applied as boundary penalty, it avoids necessity to work with subspace of FEM. Several variants of DGM exist based on computational advantages such as sparsity pattern or extension of computational stencil, complexity of numerical implementation etc.

Geometric parametrization has emerged as important application of Parametric Partial Differential Equations (PPDEs) and as alternative to shape optimization. The concept of geometric parametrization allows to transfer operator evaluated on one domain to another domain efficiently. For linear equations, this means exploiting affine parameter dependence as will be shown in later section. Model Order Reduction (MOR) on the other hand allows reducing the size of the system to be solved and working with the smaller system containing only dominant components and discarding the non-dominant components. It is pertinent to mention that identifying "dominant" components is critical to the success of model order reduction strategy. Optimization of engineering components using geometric parametrization combined with MOR for PPDEs has given quite useful results in the fields such as mechanical, naval and aeronautic designs. Also, the  faster computations obtained by MOR has helped in many query context, real time computation and quick transfer of computational results to industrial problems.

In the present work, we first introduce Discontinuous Galerkin Interior Penalty Method (DG-IPM). We subsequently introduce notion of parametrization characterizing geometry of the domain under consideration, exploit affine parameter dependence and its application in the context of offline-online decomposition. We then apply Proper Orthogonal Decomposition (POD) for constructing reduced basis space and apply Galerkin projection to project the system of equations on the space constructed by POD. Finally we present a test problem to demonstrate the introduced method and present numerical result.
 
\section{Discontinuous Galerkin formulation}
\label{DG_formulation}

Let $\Omega \subset \mathbb{R}^d$ be open bounded domain. The boundary of $\Omega$, $\partial \Omega$ is divided into Neumann boundary $\Gamma_N$ and Dirichlet boundary $\Gamma_D$ i.e. $\partial \Omega = \Gamma_N \cup \Gamma_D$. The domain $\Omega$ is divide into $N_{su}$ number of mutually non overlapping subdomains such that, $\Omega = \bigcup\limits_{i=1}^{N_{su}} \Omega_i \ , \ \Omega_i \cap \Omega_j = \emptyset \ , \ \text{for } i \neq j$. Each subdomain is divided into $nel$ number of triangular elements $\tau_k$ such that $\Omega = \bigcup\limits_{k=1}^{nel} \tau_k$. The triangulation $\mathcal{T}$ is the set of all triangular elements i.e. $\mathcal{T} = \lbrace \tau_k \rbrace_{k=1}^{nel}$. The internal boundary $\Gamma = \lbrace \partial \tau_k \rbrace_{k=1}^{nel} \backslash \partial \Omega$. We represent $\overrightarrow{n}$ as the outward pointing normal to an edge of element.

The Stokes's equation in strong form can be stated as,
\begin{gather}
-\nu \Delta \overrightarrow{u} + \nabla p = \overrightarrow{f} \ , \ \text{in } \Omega \ , \\
\nabla \cdot \overrightarrow{u} = 0 \ , \\
\overrightarrow{u} = \overrightarrow{u}_D \ , \ \text{on } \Gamma_D \ , \\
-p \overrightarrow{n} + \nu \overrightarrow{n} \cdot \nabla \overrightarrow{u} = \overrightarrow{t} \ , \ \text{on } \Gamma_N \ .
\end{gather}

The vector variable velocity $\overrightarrow{u}$ and scalar pressure $p$ are the unknowns. $\nu$ is the material property known as kinematic viscosity. Vector $\overrightarrow{f}$ is external force term or source term. $\overrightarrow{u}_D$ is the Dirichlet velocity and $\overrightarrow{t}$ is the Neumann value.

Before introducing weak form let us introduce broken Sobolev spaces for variables. 
The space for velocity is 
\begin{equation} \label{velocity_test}
\mathbb{V} = \lbrace \overrightarrow{\phi} \in (L^2(\mathcal{T}))^d| \quad \overrightarrow{\phi} \in (P^D(\tau_k))^d \quad \forall \quad {\tau_k} \in \mathcal{T} \rbrace \ .
\end{equation}
The space for pressure is 
\begin{equation} \label{pressure_test}
\mathbb{Q} = \lbrace \psi \in (L^2(\mathcal{T}))| \quad \psi \in (P^{D-1}(\tau_k)) \quad \forall \quad {\tau_k} \in \mathcal{T} \rbrace \ .
\end{equation}
Here, $P^D(\tau_k)$ denotes space of polynomials of degree at most $D$ over $\tau_k$.

In order to approximate the numerical flux we need the concept of Jump and Average operator. 
1.  For scalar quantity $p$ the jump operator is defined as,
\begin{equation}
\begin{split}
[p\overrightarrow{n}] = p^+ \overrightarrow{n}^+ + p^- \overrightarrow{n}^- \quad \textrm{on} \quad \Gamma \textrm{,}\\
[p\overrightarrow{n}] = p \overrightarrow{n} \quad \textrm{on} \quad \Gamma_D \textrm{.}
\end{split}
\end{equation}

2. For vector quantity $\overrightarrow{u}$ the jump operator is defined as,
\begin{equation}
\begin{split}
[\overrightarrow{n} \cdot \overrightarrow{u}] = \overrightarrow{n}^+ \cdot \overrightarrow{u}^+ + \overrightarrow{n}^- \cdot \overrightarrow{u}^- \quad \textrm{on} \quad \Gamma \textrm{,}\\
[\overrightarrow{n} \cdot \overrightarrow{u}] = \overrightarrow{n} \cdot \overrightarrow{u} \quad \textrm{on} \quad \Gamma_D \textrm{.}\\
\end{split}
\end{equation}

3. The average operator is defined as,
\begin{equation}\label{average operator}
\left\lbrace \overrightarrow{u} \right\rbrace = \frac{\overrightarrow{u}^+ + \overrightarrow{u}^-}{2} \textrm{.}
\end{equation} 

The weak form of Stokes equation is as follow,

\begin{equation}\label{stokes_weak_ch3}
\begin{split}
a_{IP}(\overrightarrow{u},\overrightarrow{\phi}) + b(\overrightarrow{\phi},p) + (\lbrace p \rbrace,[\overrightarrow{n} \cdot \overrightarrow{\phi}])_{\Gamma \cup \Gamma_D} = l_{IP}(\overrightarrow{\phi}) \ .
\end{split}
\end{equation}

\begin{equation}\label{a_IP}
\begin{split}
a_{IP}(\overrightarrow{u},\overrightarrow{\phi}) = (\nabla \overrightarrow{u}, \nabla \overrightarrow{\phi}) + C_{11} ([\overrightarrow{u}],[\overrightarrow{\phi}])_{\Gamma \cup \Gamma_D} \\
- \nu ({\nabla \overrightarrow{u}},[\overrightarrow{n} \otimes \overrightarrow{\phi}])_{\Gamma \cup \Gamma_D} - \nu ([\overrightarrow{n} \otimes \overrightarrow{u}],{\nabla \overrightarrow{\phi}})_{\Gamma \cup \Gamma_D} \textrm{.}
\end{split}
\end{equation}

The penalty paramter $C_{11}>0$ is an empirical constant to be kept large enough to maintain coercivity of bilinear form.

\begin{equation}\label{b}
b(\phi,\psi) = -\int_{\mathcal{T}} \psi \nabla \cdot \overrightarrow{\phi} \ ,
\end{equation}

\begin{equation}\label{l_IP}
\begin{split}
l_{IP}(\overrightarrow{\phi}) = (\overrightarrow{f},\overrightarrow{\phi}) + (\overrightarrow{t},\overrightarrow{\phi})_{\Gamma_N} + C_{11} (\overrightarrow{u}_D,\overrightarrow{\phi})_{\Gamma_D} - (\overrightarrow{n} \otimes \overrightarrow{u}_D, \nu \nabla \overrightarrow{\phi})_{\Gamma_D} \ .
\end{split}
\end{equation}

Discrete form of equations can be written in Matrix form as, 

\begin{equation} \label{Stokes_matrix_ch3}
\begin{spmatrix}{\textrm{Stiffness matrix}}
    A & B \\
    B^T & 0
\end{spmatrix}
\begin{spmatrix}{\textrm{Solution vector}}
    U \\
    P
\end{spmatrix}
=
\begin{spmatrix}{\textrm{Right hand side (Known)}}
    F_1  \\
    F_2
\end{spmatrix}
\textrm{.}
\end{equation}

Here, 
\begin{equation} \label{matrix A}
\begin{split}
A_{ij} = \sum_{k=1}^d (\frac{\partial \phi_i}{\partial x_k} , \frac{\partial \phi_j}{\partial x_k}) + \sum_{k=1}^d C_{11} ([\phi_i n_k] , [\phi_j n_k])_{\Gamma \cup \Gamma_D} \\ - \sum_{k=1}^d \nu ([\phi_i n_k] , \lbrace \frac{\partial \phi_j}{\partial x_k} \rbrace)_{\Gamma \cup \Gamma_D} - \sum_{k=1}^d \nu (\lbrace \frac{\partial \phi_i}{\partial x_k} \rbrace , [\phi_j n_k])_{\Gamma \cup \Gamma_D} \ .
\end{split}
\end{equation}

\begin{equation} \label{matrix B}
B_{ij} = - \int_{\mathcal{T}} \frac{\partial \phi_i}{\partial x_i} \psi_j +
(\lbrace \psi_j \rbrace , [n \cdot \phi_i])_{\Gamma \cup \Gamma_D} \ .
\end{equation}



\begin{enumerate}
\item{Livelihood and survival mobility are oftentimes coutcomes of uneven socioeconomic development.}
\begin{enumerate}
\item{Livelihood and survival mobility are oftentimes coutcomes of uneven socioeconomic development.}
\item{Livelihood and survival mobility are oftentimes coutcomes of uneven socioeconomic development.}
\end{enumerate}
\item{Livelihood and survival mobility are oftentimes coutcomes of uneven socioeconomic development.}
\end{enumerate}


\subparagraph{Subparagraph Heading} In order to avoid simply listing headings of different levels we recommend to let every heading be followed by at least a short passage of text. Use the \LaTeX\ automatism for all your cross-references and citations as has already been described in Sect.~\ref{sec:2}, see also Fig.~\ref{fig:2}.

For unnumbered list we recommend to use the \verb|itemize| environment -- it will automatically be rendered in line with the preferred layout.

\begin{itemize}
\item{Livelihood and survival mobility are oftentimes coutcomes of uneven socioeconomic development, cf. Table~\ref{tab:1}.}
\begin{itemize}
\item{Livelihood and survival mobility are oftentimes coutcomes of uneven socioeconomic development.}
\item{Livelihood and survival mobility are oftentimes coutcomes of uneven socioeconomic development.}
\end{itemize}
\item{Livelihood and survival mobility are oftentimes coutcomes of uneven socioeconomic development.}
\end{itemize}

\begin{figure}[t]
\sidecaption[t]
% Use the relevant command for your figure-insertion program
% to insert the figure file.
% For example, with the option graphics use
\includegraphics[scale=.65]{figure}
%
% If no graphics program available, insert a blank space i.e. use
%\picplace{5cm}{2cm} % Give the correct figure height and width in cm
%
%\caption{Please write your figure caption here}
\caption{If the width of the figure is less than 7.8 cm use the \texttt{sidecapion} command to flush the caption on the left side of the page. If the figure is positioned at the top of the page, align the sidecaption with the top of the figure -- to achieve this you simply need to use the optional argument \texttt{[t]} with the \texttt{sidecaption} command}
\label{fig:2}       % Give a unique label
\end{figure}

\runinhead{Run-in Heading Boldface Version} Use the \LaTeX\ automatism for all your cross-references and citations as has already been described in Sect.~\ref{sec:2}.

\subruninhead{Run-in Heading Boldface and Italic Version} Use the \LaTeX\ automatism for all your cross-refer\-ences and citations as has already been described in Sect.~\ref{sec:2}\index{paragraph}.

\subsubruninhead{Run-in Heading Displayed Version} Use the \LaTeX\ automatism for all your cross-refer\-ences and citations as has already been described in Sect.~\ref{sec:2}\index{paragraph}.
% Use the \index{} command to code your index words
%
% For tables use
%
\begin{table}[!t]
\caption{Please write your table caption here}
\label{tab:1}       % Give a unique label
%
% Follow this input for your own table layout
%
\begin{tabular}{p{2cm}p{2.4cm}p{2cm}p{4.9cm}}
\hline\noalign{\smallskip}
Classes & Subclass & Length & Action Mechanism  \\
\noalign{\smallskip}\svhline\noalign{\smallskip}
Translation & mRNA$^a$  & 22 (19--25) & Translation repression, mRNA cleavage\\
Translation & mRNA cleavage & 21 & mRNA cleavage\\
Translation & mRNA  & 21--22 & mRNA cleavage\\
Translation & mRNA  & 24--26 & Histone and DNA Modification\\
\noalign{\smallskip}\hline\noalign{\smallskip}
\end{tabular}
$^a$ Table foot note (with superscript)
\end{table}
%
\section{Section Heading}
\label{sec:3}
% Always give a unique label
% and use \ref{<label>} for cross-references
% and \cite{<label>} for bibliographic references
% use \sectionmark{}
% to alter or adjust the section heading in the running head
Instead of simply listing headings of different levels we recommend to let every heading be followed by at least a short passage of text.  Further on please use the \LaTeX\ automatism for all your cross-references and citations as has already been described in Sect.~\ref{sec:2}.

Please note that the first line of text that follows a heading is not indented, whereas the first lines of all subsequent paragraphs are.

If you want to list definitions or the like we recommend to use the enhanced \verb|description| environment -- it will automatically rendered in line with the preferred layout.

\begin{description}[Type 1]
\item[Type 1]{That addresses central themes pertainng to migration, health, and disease. In Sect.~\ref{sec:1}, Wilson discusses the role of human migration in infectious disease distributions and patterns.}
\item[Type 2]{That addresses central themes pertainng to migration, health, and disease. In Sect.~\ref{subsec:2}, Wilson discusses the role of human migration in infectious disease distributions and patterns.}
\end{description}

\subsection{Subsection Heading} %
In order to avoid simply listing headings of different levels we recommend to let every heading be followed by at least a short passage of text. Use the \LaTeX\ automatism for all your cross-references and citations citations as has already been described in Sect.~\ref{sec:2}.

Please note that the first line of text that follows a heading is not indented, whereas the first lines of all subsequent paragraphs are.

\begin{svgraybox}
If you want to emphasize complete paragraphs of texts we recommend to use the newly defined class option \verb|graybox| and the newly defined environment \verb|svgraybox|. This will produce a 15 percent screened box 'behind' your text.

If you want to emphasize complete paragraphs of texts we recommend to use the newly defined class option and environment \verb|svgraybox|. This will produce a 15 percent screened box 'behind' your text.
\end{svgraybox}


\subsubsection{Subsubsection Heading}
Instead of simply listing headings of different levels we recommend to let every heading be followed by at least a short passage of text.  Further on please use the \LaTeX\ automatism for all your cross-references and citations as has already been described in Sect.~\ref{sec:2}.

Please note that the first line of text that follows a heading is not indented, whereas the first lines of all subsequent paragraphs are.

\begin{theorem}
Theorem text goes here.
\end{theorem}
%
% or
%
\begin{definition}
Definition text goes here.
\end{definition}

\begin{proof}
%\smartqed
Proof text goes here.
%\qed
\end{proof}

\paragraph{Paragraph Heading} %
Instead of simply listing headings of different levels we recommend to let every heading be followed by at least a short passage of text.  Further on please use the \LaTeX\ automatism for all your cross-references and citations as has already been described in Sect.~\ref{sec:2}.

Note that the first line of text that follows a heading is not indented, whereas the first lines of all subsequent paragraphs are.
%
% For built-in environments use
%
\begin{theorem}
Theorem text goes here.
\end{theorem}
%
\begin{definition}
Definition text goes here.
\end{definition}
%
\begin{proof}
%\smartqed
Proof text goes here.
%\qed
\end{proof}
%
\begin{trailer}{Trailer Head}
If you want to emphasize complete paragraphs of texts in an \verb|Trailer Head| we recommend to
use  \begin{verbatim}\begin{trailer}{Trailer Head}
...
\end{trailer}\end{verbatim}
\end{trailer}
%
\begin{question}{Questions}
If you want to emphasize complete paragraphs of texts in an \verb|Questions| we recommend to
use  \begin{verbatim}\begin{question}{Questions}
...
\end{question}\end{verbatim}
\end{question}
\eject%
\begin{important}{Important}
If you want to emphasize complete paragraphs of texts in an \verb|Important| we recommend to
use  \begin{verbatim}\begin{important}{Important}
...
\end{important}\end{verbatim}
\end{important}
%
\begin{warning}{Attention}
If you want to emphasize complete paragraphs of texts in an \verb|Attention| we recommend to
use  \begin{verbatim}\begin{warning}{Attention}
...
\end{warning}\end{verbatim}
\end{warning}

\begin{programcode}{Program Code}
If you want to emphasize complete paragraphs of texts in an \verb|Program Code| we recommend to
use

\verb|\begin{programcode}{Program Code}|

\verb|\begin{verbatim}...\end{verbatim}|

\verb|\end{programcode}|

\end{programcode}
%
\begin{tips}{Tips}
If you want to emphasize complete paragraphs of texts in an \verb|Tips| we recommend to
use  \begin{verbatim}\begin{tips}{Tips}
...
\end{tips}\end{verbatim}
\end{tips}
\eject
%
\begin{overview}{Overview}
If you want to emphasize complete paragraphs of texts in an \verb|Overview| we recommend to
use  \begin{verbatim}\begin{overview}{Overview}
...
\end{overview}\end{verbatim}
\end{overview}
\begin{backgroundinformation}{Background Information}
If you want to emphasize complete paragraphs of texts in an \verb|Background|
\verb|Information| we recommend to
use

\verb|\begin{backgroundinformation}{Background Information}|

\verb|...|

\verb|\end{backgroundinformation}|
\end{backgroundinformation}
\begin{legaltext}{Legal Text}
If you want to emphasize complete paragraphs of texts in an \verb|Legal Text| we recommend to
use  \begin{verbatim}\begin{legaltext}{Legal Text}
...
\end{legaltext}\end{verbatim}
\end{legaltext}
%
\begin{acknowledgement}
If you want to include acknowledgments of assistance and the like at the end of an individual chapter please use the \verb|acknowledgement| environment -- it will automatically be rendered in line with the preferred layout.
\end{acknowledgement}
%
\section*{Appendix}
\addcontentsline{toc}{section}{Appendix}
%
%
When placed at the end of a chapter or contribution (as opposed to at the end of the book), the numbering of tables, figures, and equations in the appendix section continues on from that in the main text. Hence please \textit{do not} use the \verb|appendix| command when writing an appendix at the end of your chapter or contribution. If there is only one the appendix is designated ``Appendix'', or ``Appendix 1'', or ``Appendix 2'', etc. if there is more than one.

\begin{equation}
a \times b = c
\end{equation}

\bibliographystyle{spbasic.bst}
\bibliography{references.tex}
\input{references}
\end{document}



\end{document}
